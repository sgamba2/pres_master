
\documentclass{beamer}
\usepackage{caption}
\setbeamertemplate{bibliography item}{\insertbiblabel}

% Caricamento di tutti i pacchetti necessari 
\usepackage{otherResources/presentazione_allPackages}

\definecolor{mygray}{gray}{0.6}


% Settaggi documento %%
        
\hypersetup{
    pdfauthor={Nome Cognome},
    pdftitle={Titolo presentazione},
    pdfsubject={Argomento presentazione}
}
\title[Titolo breve]{Commissioning of the Mu2e tracker DAQ,\\ \vspace{1mm}planning for the Vertical Slice Test\\\vspace{1mm}and pre-pattern recognition studies}
%\subtitle{Il difficile ruolo delle navi della federazione ai confini della zona di spazio neutrale}
\institute{Università di Pisa}
\author[Sara Gamba]{\small{Sara Gamba}}
\date[21/10/24]{\small{October 21$^{\text{st}}$ 2024}}

\def\relatoreLabel{\small{\textbf{Supervisors}:}}
\def\correlatore{\small{Simone Donati (Unipi, INFN Pisa)}}
\def\relatore{\small{Pavel Murat (FNAL)}}

%\def\correlatoreLabel{Correlatore}
\def\candidatoLabel{\small{\textbf{Candidate}:}}
\def\LogoUniversita{figures/png/mu2e_logo_1.png}
\def\LogoDipartimento{otherResources/marchio_unipi_pant541.pdf}
\def\LogoFiligrana{otherResources/449099668_885883553573194_8782856259596920363_n.jpg}

%Applico tutte le personalizzazioni desiderate al tema: 
%%%%%%%%%%%%%%%%%%%%%%%%%%%%%%%%%%%%%%%%%%%%%%%%%%%%%%%%%%%
%
% Copyright 2022 by Isac Pasianotto
%
% This file may be distributed and/or modified
%
% 1. under the LaTeX Project Public License and/or
% 2. under the GNU Public License.
%
%%%%%%%%%%%%%%%%%%%%%%%%%%%%%%%%%%%%%%%%%%%%%%%%%%%%%%%%%%%

%% 	Variabili di tipo "color": 

\definecolor{bluUnits100}{rgb}{0.16,0.22,0.36} 
\definecolor{bluUnits80}{rgb}{0.22,0.3,0.51}
\definecolor{bluUnits70}{rgb}{0.25,0.35,0.58}
\definecolor{bluUnits50}{rgb}{0.41,0.51,0.74}
\definecolor{bluUnits40}{rgb}{0.56,0.63,0.81}
\definecolor{bluUnits25}{rgb}{0.78,0.82,0.9}
\definecolor{bluUnits10}{rgb}{0.93,0.94,0.96}
\definecolor{grey}{rgb}{0.3686, 0.5255, 0.6235} 
\definecolor{coolblack}{rgb}{0.0, 0.18, 0.39}
\definecolor{UBCblue}{rgb}{0.0, 0.18, 0.39} % UBC Blue (primary)
\definecolor{UBCgrey}{rgb}{0.3686, 0.5255, 0.6235} % UBC Grey (secondary)
%%	Palette di colori 

\setbeamercolor{palette primary}{bg=UBCblue,fg=white}
\setbeamercolor{palette secondary}{bg=UBCblue,fg=white}
\setbeamercolor{palette tertiary}{bg=UBCblue,fg=white}
\setbeamercolor{palette quaternary}{bg=UBCblue,fg=white}
\setbeamercolor{palette light primary}{bg=bluUnits25,fg=bluUnits100}
\setbeamercolor{palette titleframe}{bg=bluUnits10, fg=bluUnits80}

\setbeamercolor{palette primary}{bg=UBCblue,fg=white}
\setbeamercolor{palette secondary}{bg=UBCblue,fg=white}
\setbeamercolor{palette tertiary}{bg=UBCblue,fg=white}
\setbeamercolor{palette quaternary}{bg=UBCblue,fg=white}
\setbeamercolor{author in sidebar}{fg=black!20!white}
\setbeamercolor{title in sidebar}{fg=white}
\setbeamercolor{section in sidebar}{fg=white}
\setbeamercolor{structure}{fg=UBCblue} % itemize, enumerate, etc
\setbeamercolor{sidebar}{bg=UBCblue,fg=white}

		%%%%%%%%%%%%%%%%%%%%%%%%%%%%%%%%%%
		%% Impostazioni generali slide  %%
		%%%%%%%%%%%%%%%%%%%%%%%%%%%%%%%%%%

%%	Setta l'immagine da mettere come sfondo, riducendone l'opacità
%\usebackgroundtemplate{\tikz\node[opacity=0.1]{\includegraphics[height=\frameheight]{\LogoFiligrana}};}

%%	Elenchi puntati, numerati, etc.

\setbeamercolor{structure}{fg=UBCblue}
\setbeamertemplate{enumerate item}[circle]
\setbeamertemplate{itemize subitem}[ball]
% Valutare a secoda del contesto se sostituire con 
% \setbeamertemplate{items}[circle]
\setbeamercolor{alerted text}{fg=white}

%% 	Colore delle scritte nella presentazione

%\setbeamercolor{normal text}{fg=bluUnits100,bg=white}
\setbeamercolor{normal text}{fg=UBCblue}

%% 	Settaggio della linea in alto (headline)
\setbeamertemplate{navigation symbols}{}


%%	Settaggio riga in basso (footline) 

\setbeamertemplate{footline}
{
  \leavevmode%
  \hbox{%
  \begin{beamercolorbox}[wd=.333333\paperwidth,ht=2.25ex,dp=1ex,center]{author in head/foot}%
    \usebeamerfont{author in head/foot}\insertshortauthor
  \end{beamercolorbox}%
  \begin{beamercolorbox}[wd=.333333\paperwidth,ht=2.25ex,dp=1ex,center]{title in head/foot}%
   \usebeamerfont{institute in head/foot}\insertshortinstitute
  \end{beamercolorbox}%
  \begin{beamercolorbox}[wd=.333333\paperwidth,ht=2.25ex,dp=1ex,center]{date in head/foot}%
  \hspace*{12ex}
    \usebeamerfont{date in head/foot}\insertshortdate\hspace*{12ex}
    \insertframenumber{}/\inserttotalframenumber
  \end{beamercolorbox}}%
    
  \vskip0pt%
}
%%	Settaggio tittoli delle slide  

\setbeamertemplate{frametitle}{
	\begin{beamercolorbox}[wd=\paperwidth,ht=2.75ex,dp=1ex,left]{palette titleframe}
		\hspace*{2ex}\textbf{\insertframetitle}
	\end{beamercolorbox}
}


		%%%%%%%%%%%%%%%%%%%%%%%%%%%%%%%
		%% Impostazioni Prima Slide  %%
		%%%%%%%%%%%%%%%%%%%%%%%%%%%%%%%
	
	
	
	
		
\def\setTitlestyleDissertation{
	
	\defbeamertemplate*{title page}{customized}[1][]{
		
		%  Commentare il seguente ambiente {center} e decommentare {flushright} quello successivo in caso
		%	si voglia usare solo il logo dell'UNI
		
		\begin{center}
			\begin{multicols}{2}
				\includegraphics[width=0.34\textwidth]{\LogoDipartimento}
			\columnbreak
				\includegraphics[width=0.28\textwidth]{\LogoUniversita}		
			\end{multicols}
		\end{center}
	
		%	\begin{flushright}
		%		\includegraphics[width=0.45\textwidth]{\LogoUniversita}	
		%	\end{flushright}
	
		\smallskip
		
		\begin{center}		
			\usebeamerfont{title}\textbf{\inserttitle}\par
			\usebeamerfont{subtitle}\usebeamercolor[fg]{subtitle}\insertsubtitle\par
			\bigskip		
			
			%% Il seguente layout dentro l'ambiente multicols serve per le tesi.
			
			\begin{multicols}{2}
                    \begin{tabular}{c}
					\usebeamerfont{normal text}{\candidatoLabel} \\
					\usebeamerfont{author}{\insertauthor}
				\end{tabular}
    				\columnbreak
				\begin{tabular}{c}
					\usebeamerfont{normal text}{\relatoreLabel} \\
					\usebeamerfont{author}{\relatore} \\
					\usebeamerfont{author}{\correlatore}
						
				\end{tabular}					
				
			\end{multicols}
		
			\par
			
			\bigskip  	% --> nel caso di relatore e basta
			%\smallskip 	% --> nel caso di relatore + correlatore
			
			%\insertinstitute\par
			
			%\bigskip	% --> nel caso di relatore e basta
			%\smallskip	% --> nel caso di relatore + correlatore
			
			\usebeamerfont{date}\insertdate\par
			
			\bigskip	% --> nel caso di relatore e basta
			%\smallskip	% --> nel caso di relatore + correlatore
		\end{center}
	}
}


% Inizio Presentazione

\begin{document}
	
		%%%%%%%%%%%%%%%%%%%%%%%
		%%  Slide 1: TITOLO  %%
		%%%%%%%%%%%%%%%%%%%%%%%
\begin{frame}
\setTitlestyleDissertation
\maketitle
\end{frame}

		%%%%%%%%%%%%%%%%%%%%%%%%%%%%%%%%%%%%%
		%%  Slide 2: <ArgomentoPrimaSlide> %%
		%%%%%%%%%%%%%%%%%%%%%%%%%%%%%%%%%%%%%

%\section{Capitolo 1}
\begin{frame}
    \frametitle{Charged Lepton Flavour Violation}
    \vspace{-2mm}
\begin{columns}
    \begin{column}{0.7\framewidth}
        \begin{itemize}
            \item The \textbf{Standard Model} (SM) does not predict lepton flavour violation;
            \vspace{1mm}
            \item The discovery of \textbf{neutrino oscillations} prove that lepton interactions are non-diagonal in flavour; \textcolor{white}{\cite{Bernstein_2013} \cite{Kargiantoulakis_2020} \cite{universe9010054}}
            \vspace{1mm}
            \item The SM fails to explain phenomena like neutrino masses and the consequent flavour oscillations;\textcolor{white}{\cite{clfv_signorelli} \cite{bartoszek2015mu2e} \cite{bobbb} \cite{kola}}
            \vspace{1mm}
            \item The branching ratios of \textbf{CLFV}
            processes, including neutrino
            oscillations, are suppressed by factors
            proportional to ($\Delta m_\nu^2)^2 /M^4_W$ and expected to be less than $\mathcal{O}(10^{-50})$;
            \vspace{1mm}
            \item \textbf{This value is far beyond current experimental capabilities}.
        \end{itemize}
    \end{column}
    \begin{column}{0.5\framewidth}
        \begin{figure}[h]
            \centering
            \includegraphics[width=0.7\columnwidth]{figures/png/Screenshot_20240913_102556.png}
        \end{figure} 
        
        \begin{figure}[h]
            \centering
            \includegraphics[width=0.7\columnwidth]{figures/jpg/1_erkKoywyuFzJmMv4PKpc9Q.jpg}
        \end{figure} 
    \end{column}
\end{columns}
\end{frame}

		%%%%%%%%%%%%%%%%%%%%%%%%%%%%%%%%%%%%%%%%
		%%  Slide 3: <ArgomentoSecondaSlide>  %%
		%%%%%%%%%%%%%%%%%%%%%%%%%%%%%%%%%%%%%%%%

%\section{Capitolo 2}
\begin{frame}
    \frametitle{Search for CLFV}
                \vspace{-2mm}
   \begin{columns}
    \begin{column}{0.73\framewidth}
        \begin{itemize}
            \item New Physics (NP) models predict much \textbf{higher rates} of CLFV;
            \vspace{1mm}
            \item Observing CLFV would provide unambiguous evidence of \textbf{physics beyond the SM};
            \vspace{1mm}
            \item CLFV channels involving muons: $\mu^+ \rightarrow e^+ \gamma$, $\mu^- N \rightarrow e^- N$ and $\mu^+ \rightarrow e^+ e^+ e^-$;
            \vspace{1mm}
            \item $\mu^- N \rightarrow e^- N$ channel:
            \begin{itemize}
                \item Higher momentum signal and better separation from the background;
                \item Benefits from high intensity beam;
                \item Better sensitivity to CLFV in a large range of NP scenarios.
            \end{itemize}
            \item Current best limit on $\mu^- N \rightarrow e^- N$ by SINDRUM II: $R_{\mu e} < 7 \times 10^{-13}$ (90\% CL).
        \end{itemize}
    \end{column}
    \begin{column}{0.5\framewidth}
        \begin{figure}[h]
            \centering
            \hspace*{-6ex}
            \includegraphics[width=\columnwidth]{figures/png/Screenshot_20240912_093047.png}
        \end{figure}  
    \end{column}
\end{columns}
\end{frame}



        %%%%%%%%%%%%%%%%%%%%%%%%%%%%%%%%%%%%
        %% Slide 4: <ArgomentoTerzaSlide> %%
        %%%%%%%%%%%%%%%%%%%%%%%%%%%%%%%%%%%%

%\section{Capitolo 3}
\begin{frame}
    \frametitle{The Mu2e experiment}
    \vspace{-3mm}
\begin{columns}
    \begin{column}{0.71\framewidth}
        \begin{itemize}
            \item Search for neutrinoless, coherent conversion $\mu^- N \rightarrow e^- N$ in the field of an Al nucleus, by measuring: 
            $$R_{\mu e}=\frac{\mu^{-}+N(Z, A) \rightarrow e^{-}+N(Z, A)}{\mu^{-}+N(Z, A) \rightarrow \nu_\mu+N(Z-1, A)}$$
            \item Mu2e goal is to improve SINDRUM II limit by 4 orders of magnitude;
            \item The signal is a monochromatic conversion electron (CE) with energy: 
            $$ E_{CE} = m_\mu - E_{recoil} - E_{bind} = 104.97 \ \text{MeV}$$
            where $m_\mu$ is the muon mass, $E_{recoil}$ the target nucleus recoil energy and $E_{bind}$ the muonic atom $1s$ state binding energy.
        \end{itemize}
    \end{column}
    \begin{column}{0.5\framewidth}
        \begin{figure}[h]
            \centering
            \includegraphics[width=0.9\columnwidth]{figures/png/Screenshot_20240913_161505.png}
        \end{figure} 
                \begin{figure}[h]
            \centering
            \includegraphics[width=0.75\columnwidth]{figures/png/Screenshot_20240913_160115.png}
        \end{figure} 
    \end{column}
\end{columns}
\end{frame}


\begin{frame}
    \frametitle{Background sources}
            \vspace{-3mm} 

         \begin{figure}[h]
            \centering
            \hspace*{-4ex}
            \includegraphics[width=1.\framewidth]{figures/png/Screenshot_20240225_102708.png}
        \end{figure}
        \setlength{\leftmargini}{-0.5em}
        \vspace{-3mm} 
      
    \begin{itemize}
      {\small
    \item \textbf{Cosmics} $\rightarrow$ veto enclosing the detector {\footnotesize($\approx$0.046 evs/RunI)};   
\item \textbf{Intrinsic} $\rightarrow$ 1 MeV/c momentum resolution:
\begin{itemize}
 \item Decay In Orbit $\mu^- N \rightarrow e^- \bar{\nu}_e\nu_\mu N $ {\footnotesize($\approx$0.038 evs/RunI)};
 \item Radiative Muon Capture $\mu^- N \rightarrow\gamma \nu_\mu N'^* $ {\footnotesize($<0.0024$ evs/RunI)}.
\end{itemize}
\item \textbf{Delayed processes from $\bar{p}$} $\rightarrow$ absorbers in the TS {\footnotesize($\approx0.010$ evs/RunI)};
\item \textbf{Prompt processes} $\rightarrow$ pulsed beam + delayed
live window:
\begin{itemize}
    \item Radiative Pion Capture $\pi^- N \rightarrow \gamma N' ^*$ {\footnotesize($\approx0.010$ evs/RunI)};
 \item $\pi$ and $\mu$ Decay In Flight {\footnotesize($<2\times 10^{-3}$ evs/RunI)};
 \item Beam electrons {\footnotesize($<1\times 10^{-3}$ evs/RunI)}.
     \end{itemize}
       }
      \end{itemize}
\end{frame}

\begin{frame}
    \frametitle{The Mu2e experimental setup}
    \vspace{-7mm}
     \begin{figure}[h]
            \centering
            \hspace*{-4ex}
            \includegraphics[width=1.1\framewidth, height=3.5cm ]{figures/png/Screenshot_20240913_162936.png}
        \end{figure}
                \setlength{\leftmargini}{-0.5em}
\vspace{-2mm}
        \begin{itemize}
        
            \item \textbf{Production Solenoid}:
            \begin{itemize}
                \item 8 GeV pulsed proton beam interacts with the W target and mostly $\pi$s are produced;
                \item graded field for backward collection.
            \end{itemize}
            \item \textbf{Transport Solenoid}:
            \begin{itemize}
                \item it allows for $\pi$ decay and $\mu$ transport;
                \item $S$-shape for charged particle selection;
                \item it selects muons with $p \lesssim 100 $ MeV/c;
                \item rotating collimator COL3 selects $\mu^-$ or $\mu^+$ beam.
            \end{itemize}
\item \textbf{Detector Solenoid}: Stopping Target, $p$ absorber and detectors.
        \end{itemize}
\end{frame}
\begin{frame}
    \frametitle{The electromagnetic calorimeter}
    \begin{columns}
            \begin{column}{0.65\framewidth}
      Calorimeter is vital for:
      \begin{itemize}
          \item \textbf{PID} ($E/p$);
          \item Seed for \textbf{track reconstruction};
          \item Fast online \textbf{trigger} filter.
      \end{itemize}
      \textbf{Design}:
      \begin{itemize}
        \item 2 hollow disks of crystals, 70 cm apart;
          \item 2 $\times$ 674 CsI crystals per disk, each coupled to 2 SiPMs.
      \end{itemize}
      \textbf{Performance}:
      \begin{itemize}
          \item $\sigma_E/E \sim$ 10\%;
          \item $\sigma_{xy}\sim$ 6 mm;
          \item $\Delta t < 500$ ps.
      \end{itemize}
               \end{column}
        \begin{column}{0.5\framewidth}
            \begin{figure}[h]
            \centering
            \includegraphics[width=\columnwidth ]{figures/png/Screenshot_20240322_121000.png}
        \end{figure}
              \begin{figure}[h]
            \centering
            \includegraphics[width=\columnwidth ]{figures/png/Screenshot_20240706_151533.png}
        \end{figure}
        \end{column}
    \end{columns}
\end{frame}

\begin{frame}
    \frametitle{The straw tracker}
    \begin{columns}
    \begin{column}{0.65 \framewidth}
\textbf{Purpose}:
\begin{itemize}
\item momentum measurement with $\Delta p<300$ keV/c FWHM + 950 keV/c energy losses (ST and proton absorber) (DIO). 
\end{itemize}
\textbf{Design}:
\begin{itemize}
    \item 3 m downstream of the ST (B$\sim$1 T);
    \item hollow geometry (low $p_T $ particles);
    \item 3 m long tracker in vacuum;
    \item 96 straws per panel, 6 panels per plane, 2 planes per station;
        \item 18 tracking stations: 216 panels;
    \item 5 mm diameter and 40-110 cm long straws filled with a 80\%:20\% Ar:CO$_2$ mixture at a pressure of 1 atm.
       
        
        
    \end{itemize}
    
    \end{column}
    \begin{column}{0.5 \framewidth}
    \vspace{-10mm}
      \begin{figure}[H]
          \centering
\includegraphics[width=0.8\columnwidth]{figures/png/Screenshot_20240306_222803.png}
          \label{fig:enter-label} 
      \end{figure}
     \begin{figure}[H]
          \centering
            \includegraphics[width=0.8\columnwidth]{figures/png/Screenshot_20240706_163056.png}
          \label{fig:enter-label} 
      \end{figure}
    \end{column}
    \end{columns}
    \end{frame}
\begin{frame}
    \frametitle{The tracker readout and DAQ}
    \vspace{-6mm}
    \begin{columns}
         \begin{column}{0.6\framewidth}
         \setlength{\leftmargini}{1em}
         \begin{itemize}
         {\small
         \item Signal is readout from both ends by  \textbf{preamps} (\textbf{CAL} and \textbf{HV} side);
         \item Analog signals are sent to the \textbf{DRAC} {\footnotesize(Digitizer Readout \& Assembler Controller)} and processed by 2 \textbf{TDC}s and one \textbf{ADC};
        \item The 2 digi-FPGA create one data packet for each hit containing the \textbf{two hit times} and \textbf{one waveform};
        \item Data packets are transferred to the \textbf{ROC} {\footnotesize(Readout Controller)};
        \item ROC collects, buffer and transfer data from digi-FPGAs to \textbf{DTC} {\footnotesize(Data Transfer Controller)} installed on \textbf{DAQ} computers;
        \item DTC sends data request to the ROC and data from DTC is sent to the Event Builder.}
         \end{itemize}
    \end{column}
    \begin{column}{0.5\framewidth}
           \begin{figure}[h]
          \centering
                    \hspace*{-1.2em}
            \includegraphics[width=1.1\columnwidth]{figures/png/Screenshot_20240919_110354.png}
          \label{fig:enter-label} 
      \end{figure} 
           \begin{figure}[h]
          \centering
          \vspace{-10mm}
            \includegraphics[width=1.02\columnwidth]{figures/png/Screenshot_20240529_133230.png}
          \label{fig:enter-label} 
      \end{figure} 
    \end{column}
    \end{columns}

\end{frame}
\begin{frame}
    \frametitle{My Thesis}
    
\begin{columns}
   \begin{column}{1.15\framewidth} 
         \setlength{\leftmargini}{1.3em}
\vspace{-4mm}
\begin{itemize}
\item Mu2e is starting detector \textbf{commissioning} and \textbf{calibration};
     \vspace{2mm}
\item My work consists of a \textbf{comprehensive study of the Mu2e tracker}:
\vspace{1.8mm}
  \begin{itemize}
    \item \textbf{Vertical Slice Test} (VST). The entire 
testing chain, from the straws to the readout, to processed data on disk:
     \vspace{1mm}
 \begin{itemize}
     \item initial tracker \textbf{DAQ} and \textbf{FEE} testing;
          \vspace{1mm}
     \item validation of the \textbf{ROC} readout and buffering;
          \vspace{1mm}
     \item study of tracker \textbf{preamps} performance.
     \end{itemize}
     \vspace{1.8mm}

    \item First steps towards the tracker timing \textbf{calibration} with \textbf{cosmics}:
         \vspace{1mm}
    \begin{itemize}
            \item determine \textbf{signal propagation} and \textbf{channel-to-channel delay};
                 \vspace{1mm}
            \item develop an \textbf{unbiased cosmic track reconstruction} procedure.
    \end{itemize}
         \vspace{1.8mm}

    \item \textbf{Mu2e Offline}. Pre-pattern recognition studies:
         \vspace{1mm}
    \begin{itemize}
        \item estimated data volume for Mu2e data-taking 
is >\textbf{7 PBytes/year};
     \vspace{1mm}
\item the \textbf{primary source} of hits in the 
Mu2e tracker will be $\delta$-electrons;
     \vspace{1mm}
\item important to identify those hits without losing \textbf{CE efficiency}.
    \end{itemize}
        \end{itemize}
\end{itemize}
 \end{column}
 \end{columns}
     

\end{frame}
\begin{frame}
    \frametitle{Outline}
    
\begin{itemize}
\item Commissioning of the tracker DAQ and FEE:
\begin{itemize}
         \vspace{2mm}

    \item validation of ROC readout;
             \vspace{1.5mm}

    \item \textcolor{mygray}{study of preamplifiers performance}.
\end{itemize}
\vspace{4mm}
    \item \textcolor{mygray}{First steps towards the station calibration;}
    \vspace{6mm}

    \item \textcolor{mygray}{Pre-pattern recognition studies;}
   \vspace{6mm}

    \item \textcolor{mygray}{Conclusions.}
\end{itemize}
\end{frame}

 \begin{frame}{Description of test stand setup}
 \vspace{-3mm}
\begin{columns}
    \begin{column}{0.69\framewidth}
             \setlength{\leftmargini}{1.3em}
      \begin{itemize}
      \item \textbf{ROC readout validation};
      \vspace{1mm}
          \item \textbf{TS1 tracker test stand}: one DTC connected to the DAQ computer, one ROC (one tracker panel-96 channels);
             \vspace{1mm}
          \item ROC can be operated in two different data readout modes:
          \begin{itemize}
              \item MODE 1: emulated data readout mode;
              \item \textbf{MODE 2}: digi-FPGA readout mode.
          \end{itemize}
             \vspace{1mm}
          \item digi-FPGAs pulsed by their \textbf{internal pulser} at $f_{gen}$= 250 kHz or 60 kHz;
             \vspace{1mm}
          \item \textbf{Event Window} ($T_{EW}$): the time interval between two proton pulses, varied between 700 ns to 50 $\mu$s;
             \vspace{1mm}
          \item The ROC firmware has an internal \textbf{hit buffer} which stores up to \textbf{255 hits}.

      \end{itemize}
          \end{column}
\begin{column}{0.45\framewidth}
        \begin{figure}[h]
          \centering
            \includegraphics[width=0.8\columnwidth]{figures/png/Screenshot_20240712_102528.png}
          \label{fig:enter-label} 
      \end{figure} 
\end{column}
\end{columns}

          \end{frame}




\begin{frame}{Logic of data taking}
\begin{columns}
   \begin{column}{1.15\framewidth} 
         \setlength{\leftmargini}{1.3em}
         \vspace{-3mm}
        \begin{itemize}
         \item Depending on $T_{gen}=1/f_{gen}$ and $T_{EW}$, the data taking can proceed in two different modes:
  \begin{itemize}
  \item $N_{gen}\geq255$: $N_{readout}=255$;
    \item $N_{gen}<255$: $N_{readout}<255$;
  \end{itemize}
      \item Each FPGA has its own generator and pulses from different generators are offset ($\in [0 ,T_{gen}]$) with respect to each other;
      \item Timing of generator pulses uncorrelated with the beginning of the EW $\rightarrow$ different number of hits in an EW;
      \item Offsets between channels (same digi-FPGA) are about few ns and can be measured;
      \item Channel readout sequence is fixed.
  \end{itemize}
  \vspace{-4mm}
              \begin{figure}[H]
          \centering
          \includegraphics[width=0.8 \framewidth]{figures/png/finalimg.png}
          \label{fig:enter-label} 
      \end{figure}
   \end{column}
   \end{columns}
    \end{frame}


    \begin{frame}{Monte Carlo simulation}
                 \vspace{-3mm}
    \begin{columns}
  \begin{column}{0.65\framewidth} 
        \begin{itemize}
        {\small
            \item ROC readout logic emulated with a bit-level C++ simulation;
            \item Simulated parameters:
            \begin{itemize}
                \item number of hits in each channel;
                \item number of readout hits per event.
            \end{itemize}
            \item digi-FPGAs and channel to channel offsets considered. }
        \end{itemize}
       {\small  \textbf{Steps of the simulation:}}
\begin{itemize}
 {\small
\item EW starts at $t=0$ s;
  \item The 1st pulse is generated $T_0\in [0 ,T_{gen}]$;
    \item Next pulses: $T_i = T_{i-1} + T_{gen}$, until $T_i> T_{EW}$;
\item Pulses are generated in each channel following the readout sequence;
  \item The procedure \textit{continues} until all hits 
  have been $readout$, or $N_{hits}>255$. }
\end{itemize}
    \end{column}
     \begin{column}{0.5\framewidth} 
      \begin{figure}[H]
          \centering
      \includegraphics[width=\columnwidth]{figures/png/Screenshot from 2023-12-03 11-50-50.png}
      \label{fig:delay1}
    \end{figure}
    \begin{figure}[H]
          \centering
      \includegraphics[width=\columnwidth]{figures/png/Screenshot from 2023-12-03 11-50-33.png}

      \label{fig:delay2}
    \end{figure}
       \end{column}
      
   \end{columns}
    \end{frame}


\begin{frame}{Hit timing distribution: overflow mode}
\vspace{-4mm}
\begin{columns}
    \begin{column}{0.5\framewidth}
        \begin{figure}[H]
          \centering
          \hspace*{-2em}
        \includegraphics[width=1.\columnwidth]{figures/pdf/figure_00007_timedistr_roc_simulation_ch0_281.pdf}
          \label{fig:enter-label} 
          \end{figure}
    \end{column}
    \begin{column}{0.5\framewidth}
        \begin{figure}[H]
          \centering
                    \hspace*{-1em}
        \includegraphics[width=1.\columnwidth]{figures/pdf/figure_00003_timedistr_roc_simulation_ch2_281.pdf}
          \label{fig:enter-label} 
          \end{figure} 
    \end{column}
\end{columns}
 \begin{columns}
    \begin{column}{1.15\framewidth}
    \vspace{0mm}

     \begin{itemize}
    
     \item The distribution of the hit time for channel 0 of digi-FPGA-1 (Left) and channel 2 of digi-FPGA-2 (Right);
     \item $T_{EW}$ = 50 $\mu$s and $f_{gen}$ = 60 kHz;
     \item Left distribution is uniform, Right one is non-trivial;
     \item Different behaviour for different channels in different FPGAs;
        \item Apparently there are interruptions of channel 2 in the second FPGA;
        \item Everything can be explained with the $occupancy$ plot.
    \end{itemize}
         \end{column}
\end{columns}   
\end{frame}
\begin{frame}{Hit timing distribution: regular mode}
\vspace{-10mm}
\begin{columns}
    \begin{column}{0.5\framewidth}
        \begin{figure}[H]
          \centering
          \hspace*{-2em}
        \includegraphics[width=1.2\columnwidth]{figures/pdf/figure_00001_timedistr_roc_simulation_10538.pdf}
          \label{fig:enter-label} 
          \end{figure}
    \end{column}
    \begin{column}{0.5\framewidth}
        \begin{figure}[H]
          \centering
                    \hspace*{-1em}
        \includegraphics[width=1.2\columnwidth]{figures/pdf/figure_00012_timedistr_roc_simulation_ch2_105038.pdf}
          \label{fig:enter-label} 
          \end{figure} 
   \end{column}
\end{columns}
 \begin{columns}
    \begin{column}{1.15\framewidth}
     \begin{itemize}
      \item The distribution of the hit time for channel 0 of digi-FPGA-1 (Left) and channel 2 of digi-FPGA-2 (Right);
          \item $T_{EW}$ = 25 $\mu$s and $f_{gen}$ = 60 kHz;
        \item Same behaviour for different channels in different FPGAs;
        \item No interruptions of channel 2 in the second FPGA.    \end{itemize}
             \end{column}
\end{columns}     
         
\end{frame}



\begin{frame}{$Occupancy$ plots}
\vspace{-4.4mm}
\begin{columns}
    \begin{column}{0.5 \framewidth}
    \begin{figure}[H]
          \centering
          \hspace*{-2em}
        \includegraphics[width=1.1\columnwidth]{figures/pdf/figure_00004_nhitsvschannel_roc_simulation_281.pdf}
          \label{fig:dfjkdsfh} 
\end{figure}    
    \end{column}
    \begin{column}{0.5 \framewidth}
           \begin{figure}[H]
          \centering
          \hspace*{-2em}
        \includegraphics[width=1.1\columnwidth]{figures/pdf/figure_00002_nhitsvschannel_roc_simulation_2.pdf}
          \label{fig:dfjkdsfh} 
\end{figure} 
    \end{column}
\end{columns}      
\vspace{-3.6mm}
 \begin{columns}
    \begin{column}{1.17\framewidth}
     \begin{itemize}
     {\small
      \item $Occupancy$ plot: number of hits versus channel number (data red, MC blue);
      \item The bin ordering corresponds to the channel readout ordering;
    \item Overflow mode (Left):  }
    \begin{itemize}
    {\small
        \item channels 0-68: 48 digi-FPGA-1 channels with 4 hits (192 hits) and 21 digi-FPGA-2 channels with 3 hits (63 hits);
        \item channels 0-75: 48 digi-FPGA-1 channels with 3 hits (144 hits) and 27 digi-FPGA-2 channels with 4 hits (108 hits) and 1 with 3 hits (111 hits);
        \item channels 0-85: 48 digi-FPGA-1 channels with 3 hits (144 hits) and 37 digi-FPGA-2 channels with 3 hits (111 hits).}
    \end{itemize}
    \item  {\small Regular mode (Right): all channels with same occupancy ($N_{hits}<255$).
    }
   
        \end{itemize}
             \end{column}
\end{columns}     
\end{frame}


\begin{frame}{Number of hits distribution}
\vspace{-10mm}
\begin{columns}
    \begin{column}{0.5\framewidth}
    \hspace{-30mm}
        \begin{figure}[H]
          \centering
          \hspace*{-2em}
        \includegraphics[width=1.2\columnwidth]{figures/pdf/figure_00008_nhits_281.pdf}
          \label{fig:ov} 
          \end{figure}
    \end{column}
    \begin{column}{0.5\framewidth}
        \begin{figure}[H]
          \centering
          \hspace*{-1em}
        \includegraphics[width=1.2\columnwidth]{figures/pdf/figure_00009_nhits_105038.pdf}
          \label{fig:un} 
          \end{figure} 
    \end{column}
\end{columns}
 \vspace{-3mm}
 \begin{columns}
    \begin{column}{1.15\framewidth}
     \begin{itemize}
        \item Overflow mode (Left): distribution of number of hits peaked in 255; 
        \item Regular mode (Right): the number of hits  distribution depends on the relative offset of the EW  with respect to the digi-FPGA pulsers and it varies from 
144 to 192;
            \item Agreement between MC and data at a level of $10^{-3}$.

    \end{itemize}  
    \end{column}
    \end{columns}
\end{frame}





\begin{frame}
    \frametitle{Outline}
    
\begin{itemize}
\item Commissioning of the tracker DAQ and FEE:
\begin{itemize}
         \vspace{2mm}

    \item \textcolor{mygray}{validation of ROC readout;}
             \vspace{1.5mm}

    \item study of preamplifiers performance.
\end{itemize}
\vspace{4mm}
    \item \textcolor{mygray}{First steps towards the station calibration;}
    \vspace{6mm}

    \item \textcolor{mygray}{Pre-pattern recognition studies;}
    \vspace{6mm}

    \item \textcolor{mygray}{Conclusions.}
\end{itemize}
\end{frame}




\begin{frame}
    \frametitle{Test 1: channel occupancy versus channel ID}
\vspace{-3mm}
\begin{columns}
\begin{column}{1.15\framewidth}
    \setlength{\leftmargini}{1.1em}
 \begin{itemize}
 {\small
     \item Same test stand: 1 or 2 ROCs and one DTC, plus preamps on the CAL side;
     \item CAL side digi-FPGA generates calibration pulses, pulsing every 8th channel across 12 RUNs (different starting channel);
          \item The frequency was set to 50 kHz and  $T_{EW}=$50 $\mu$s, 2 or 3 hits per channel;
     \item Looking for cross talks, non-uniform occupancy, dead channels.
   }
 \end{itemize}
 \end{column}
  \end{columns}
    \vspace{-2mm}
   \begin{columns}
 \begin{column}{0.5\framewidth}
     \begin{figure}[!h]
      \centering
      \includegraphics[width=\columnwidth]{figures/pdf/run105421_nh_vs_ch.pdf}
     \label{fig:normalhits}
\end{figure}
 \begin{itemize}
 \end{itemize}
 \end{column}
  \begin{column}{0.5\framewidth}
     \begin{figure}[!h]
      \centering
      \vspace{-3mm}
      \includegraphics[width=\columnwidth]{figures/pdf/run105346_nh_vs_ch.pdf}
     \label{fig:normalhits}
\end{figure}
 \end{column}
\end{columns}
 \vspace{-10mm}
 \begin{columns}
 \begin{column}{1.15\framewidth}
     \setlength{\leftmargini}{1.1em}
 \begin{itemize}
 \item (Left): regular occupancy;
    \item (Right): 94th channel dead (preamp substituted) and $N_{hits}>3$ in some channels $\rightarrow$ time distribution and inverted waveforms in Test 2.
    \end{itemize}
     \end{column}
\end{columns}
\end{frame}

\begin{frame}
\frametitle{Test 1: channel occupancy versus channel ID}
\vspace{-4mm}
\begin{columns}
\begin{column}{0.5\framewidth}
         \begin{figure}[!h]
      \centering
      \hspace*{-2em}
      \includegraphics[width=1.1\columnwidth]{figures/pdf/run105420_nh_vs_ch.pdf}
     \label{fig:normalhits}
\end{figure}
\end{column}
\begin{column}{0.5\framewidth}
      \begin{figure}[!h]
      \centering
            \hspace*{-1em}
\includegraphics[angle=90,width=1.1\columnwidth]{figures/jpg/photo_6028424923279639562_y.jpg}
     \label{fig:normalhits}
\end{figure}
\end{column}
\end{columns}
\begin{columns}
\begin{column}{1.15\framewidth}
    \setlength{\leftmargini}{1.2em}
 \begin{itemize}
  \item (Left): occupancy plot with cross talks in first odd channels and only asymmetric (e.g. 3$\rightarrow$5, not seen 3$\rightarrow$1); 
\item (Right): preamp boards are mounted vertically and odd channels are those on the PCB board;
\item The distance between the first channels is slightly lower;
\item The solution to these cross talks is still object of study.
 \end{itemize}
\end{column}
\end{columns}
\end{frame}




\begin{frame}
    \frametitle{Test 2: analysis of the readout pulses waveforms}
    \vspace{-3mm}
    \begin{columns}
\begin{column}{1.15\framewidth}
    \setlength{\leftmargini}{1.2em}
 \begin{itemize}
 {\small
 \item Same test stand;
  \item Checking signal uniformity among channels within the same ROC or across multiple ROCs, and among different events;
  \item 40 MHz ADC (25 ns sample width) and pulser frequency set to 50 kHz.}
  \end{itemize}
    \end{column}
    \end{columns}
    \vspace{-2mm}
    \begin{columns}
\begin{column}{0.5\framewidth}
         \begin{figure}[!h]
      \centering
      \hspace*{-2em}
      \includegraphics[width=1.\columnwidth]{figures/pdf/wf_ch50_0.pdf}
     \label{fig:normalhits}
\end{figure}
\end{column}
\begin{column}{0.5\framewidth}
      \begin{figure}[!h]
      \centering
            \hspace*{-1em}
\includegraphics[width=1.\columnwidth]{figures/png/baseline_ch00.png}
     \label{fig:normalhits}
\end{figure}
\end{column}
\end{columns}
\vspace{-2mm}
    \begin{columns}
\begin{column}{1.15\framewidth}
    \setlength{\leftmargini}{1.2em}
 \begin{itemize}
 {\small
 \item (Left): regular waveform;
\item Flat distribution in the first 10 samples (baseline), high positive charge peak with a sharp leading edge, negative tail;
\item (Right): fitted baseline distibution, with mean at 210 ADC counts and FWHM$=2 \sqrt{2 \text{ln}2}\sigma\sim$4.5 ADC counts. }
\end{itemize}
\end{column}
\end{columns}
\end{frame}

 
\begin{frame}
    \frametitle{Test 2: analysis of the readout pulses waveforms}
    \vspace{-4mm}
    \begin{columns}
\begin{column}{1.15\framewidth}
    \setlength{\leftmargini}{1.2em}
 \begin{itemize}
{\small \item Different baseline values indicating noise, dips, inverted waveforms.}
  \end{itemize}
    \end{column}
    \end{columns}
        \vspace{-3mm}
    \begin{columns}
\begin{column}{0.5\framewidth}
         \begin{figure}[!h]
      \centering
      \hspace*{-2em}
      \includegraphics[width=\columnwidth]{figures/pdf/wf_ch58_1.pdf}
     \label{fig:normalhits}
\end{figure}
\end{column}
\begin{column}{0.5\framewidth}
      \begin{figure}[!h]
      \centering
            \hspace*{-1em}
\includegraphics[width=\columnwidth]{figures/pdf/wf_ch50_1.pdf}
     \label{fig:normalhits}
\end{figure}
\end{column}
\end{columns}
\vspace{-1.5mm}
    \begin{columns}
    \begin{column}{0.63\framewidth}
        \setlength{\leftmargini}{1.1em}
      \begin{itemize}
 {\small
 \item (Left): dips of specific depths (64, 128 or 192) $\rightarrow$ ADC 6th or 7th bits;
\item Problematic samples identified and excluded from the baseline estimate;
\item (Right): inverted waveform $\rightarrow \Delta t $ distribution peaked in 16 $\mu$s (regular) and 4 $\mu$s (inverted). Trigger on trailing edge of 4 $\mu$s long input pulses.}

\end{itemize}
\end{column}
\begin{column}{0.5\framewidth}
         \begin{figure}[!h]
      \centering
      \hspace*{-2em}
      \includegraphics[width=\columnwidth]{figures/png/deltathits.png}
     \label{fig:normalhits}
\end{figure}
\end{column}
\end{columns}
\end{frame}


\begin{frame}
    \frametitle{Test 2: analysis of the readout pulses waveforms}
    \vspace{-4mm}
    \begin{columns}
\begin{column}{1.15\framewidth}
    \setlength{\leftmargini}{1.2em}
 \begin{itemize}
{\small \item (Left): pulse height (PH) (charge) distribution with 2/3 peaks.}
  \end{itemize}
    \end{column}
    \end{columns}
        \vspace{-3mm}
    \begin{columns}
\begin{column}{0.5\framewidth}
         \begin{figure}[!h]
      \centering
      \hspace*{-2em}
      \includegraphics[width=0.9\columnwidth]{figures/pdf/pulseheight.pdf}
     \label{fig:normalhits}
\end{figure}
\end{column}
\begin{column}{0.5\framewidth}
      \begin{figure}[!h]
      \centering
            \hspace*{-1em}
\includegraphics[width=0.9\columnwidth]{figures/pdf/tmean1.pdf}
     \label{fig:normalhits}
\end{figure}
\end{column}
\end{columns}
\vspace{-5mm}
    \begin{columns}
    \begin{column}{0.55\framewidth}
        \setlength{\leftmargini}{1.em}
        \vspace{-2mm}
      \begin{itemize}
 {\small
\item (Top Right): $s_{mean} = \frac{\sum_i \text{sample}_i \cdot q_i }{\sum_i q_i} $ distribution, correlated with PH (charge) peaks;
\item (Bottom Right): simulation of the charge and PH distribution behaviour;
\item This is an artifact of the pulser 
timing shifted with respect to the ADC 
clock of few ns.}

\end{itemize}
\end{column}
\begin{column}{0.55\framewidth}
         \begin{figure}[!h]
      \centering
      \hspace*{-2em}
      \includegraphics[width=1.1\columnwidth]{figures/png/pres.png}
     \label{fig:normalhits}
\end{figure}
\end{column}
\end{columns}
\end{frame}


\begin{frame}
\frametitle{Test 2: analysis of the readout pulses waveforms}
        \vspace{-4mm}
    \begin{columns}
\begin{column}{1.15\framewidth}
    \setlength{\leftmargini}{1.2em}
 \begin{itemize}
{\small \item Charge distribution used to check noisy channels (Left) and glitches (Right).}
  \end{itemize}
    \end{column}
    \end{columns}

     \vspace{-3mm}
    \begin{columns}
\begin{column}{0.5\framewidth}
         \begin{figure}[!h]
      \centering
      \hspace*{-1em}
    \includegraphics[width=0.95\columnwidth]{figures/pdf/noise.pdf}
     \label{fig:normalhits}
\end{figure}
\end{column}
\begin{column}{0.5\framewidth}
      \begin{figure}[!h]
      \centering
            \hspace*{-1em}
\includegraphics[width=0.95\columnwidth]{figures/pdf/glitch.pdf}
     \label{fig:normalhits}
\end{figure}
\end{column}
\end{columns}
\vspace{-3mm}
    \begin{columns}
    \begin{column}{1.15\framewidth}
        \setlength{\leftmargini}{1.em}
      \begin{itemize}
 {\small
\item Check of the response uniformity
across channels.}
\end{itemize}
\end{column}
\end{columns}
\vspace{-3mm}
\begin{columns}
\begin{column}{0.5\framewidth}
         \begin{figure}[!h]
      \centering
      \hspace*{-1em}
      \includegraphics[width=1.\columnwidth]{figures/pdf/q_vs_ch1.pdf}
     \label{fig:normalhits}
\end{figure}
\end{column}
\begin{column}{0.5\framewidth}
         \begin{figure}[!h]
      \centering
      \hspace*{-1em}
      \includegraphics[width=1.\columnwidth]{figures/pdf/ph_vs_ch1.pdf}
     \label{fig:normalhits}
\end{figure}
\end{column}
\end{columns}
\end{frame}



\begin{frame}
    \frametitle{Outline}
    
\begin{itemize}
\item \textcolor{mygray}{Commissioning of the tracker DAQ and FEE:}
\begin{itemize}
         \vspace{2mm}

    \item \textcolor{mygray}{validation of ROC readout;}
             \vspace{1.5mm}

    \item \textcolor{mygray}{study of preamplifiers performance}.
\end{itemize}
\vspace{4mm}
    \item First steps towards the station calibration;
    \vspace{6mm}

    \item \textcolor{mygray}{Pre-pattern recognition studies;}
  \vspace{6mm}
    \item \textcolor{mygray}{Conclusions.}
\end{itemize}
\end{frame}

\begin{frame}
    \frametitle{First steps towards the station calibration}
    \vspace{-3mm}
    \begin{columns}
        \begin{column}{1.15\framewidth}
                        \setlength{\leftmargini}{1.2em}
            \begin{itemize}
                {\small    
             \item \textbf{Calibration goal}: straw longitudinal position resolution $\lesssim$4 cm;}
            \end{itemize}
        \end{column}
    \end{columns}
        \vspace{-1.5mm}
    \begin{columns}
        \begin{column}{0.65\framewidth}
                \setlength{\leftmargini}{1.2em}
            \begin{itemize}
             {\small 
             \item TDCs measure arrival times $t_1$ and $t_2$;

                \item $v$: \textbf{signal propagation velocity};
    \item $x_{\text{track}}$: \textbf{reconstructed track position along the wire};
    
                \item $t_0$ particle crossing, $t_d$ drift time, $L$ straw length, $d_i$ delays by FEE;
                                          
                }   
            \end{itemize}
        \end{column}
        \begin{column}{0.48\framewidth}
            \begin{equation*}
\begin{aligned}
    t_1 &= t_0 + \frac{x_{\text{track}}}{v} + t_d + d_1 \\
    t_2 &= t_0 + \frac{L - x_{\text{track}}}{v} + t_d + d_2 \\
    \Delta & t_{12} = \frac{2x_{\text{track}}-L}{v} +(d_1-d_2)
\end{aligned}
\end{equation*}
\vspace{-3mm}
        \end{column}
    \end{columns}
    \vspace{-1.5mm}
    \begin{columns}
        \begin{column}{1.16\framewidth}
                        \setlength{\leftmargini}{1.28em}
            \begin{itemize}
               {\small 
                \item \textbf{Calibration}: $v$ from $\Delta t_{12}$ (TDCs) correlated with $x_{\text{track}}$ (\textbf{unbiased});
                \vspace{-0.5mm}
               \item $x_{track}$ determined by straw \textbf{"yes or no"} information $\rightarrow$ \textbf{station geometry}.}
            \end{itemize}
        \end{column}
    \end{columns}
    \begin{columns}
        \begin{column}{0.5 \framewidth}
         \vspace{-3mm}
            \begin{figure}[!h]
    \centering
    \includegraphics[width =0.8\columnwidth]{figures/png/gassystem.png}
    \label{fig:gassystem}
\end{figure}
        \end{column}
        \begin{column}{0.65 \framewidth}
        \vspace{-0.5mm}
                                \setlength{\leftmargini}{0.3em}
            \begin{itemize}
               {\small 
               \item First calibration with \textbf{cosmics};
               \vspace{-0.5mm}
               \item Unbiased reconstruction with $horizontal$ $orientation$; 
                              \vspace{-0.5mm}
               \item \textbf{Operational constraints}: gas system (sealing with vertical valves), space, fragility. Designed to be operated \textbf{vertically};
                 \vspace{-0.5mm}
               \item \textbf{Simulation with vertical station to assess biases and feasibility}.}
            \end{itemize}
        \end{column}
    \end{columns}
\end{frame}


\begin{frame}
    \frametitle{Monte Carlo muon selection and reconstruction}
\vspace{-4.6mm}
    \begin{columns}
\begin{column}{0.65\framewidth}
   \setlength{\leftmargini}{1.2em}
      \begin{itemize}
      {\small \item \textbf{Cosmics as calibration source}:}
      \vspace{-1mm}
      \begin{itemize}
    {\small \item standard detector operations;
    \item flux is $\sim 1 \ \text{cm}^{-2} \text{min}^{-1}$ 
    (for horizontal detectors) and $E_{mean}\sim$ GeV;
    \item MIP;
    \item $v_{\mu}\sim c \rightarrow$ align channel offsets.}
\end{itemize}
{\small \item \textbf{Straw information}:}
\vspace{-1mm}
\begin{itemize}
    {\small \item the direction $(D_{x,i},D_{y,i})$;
    \item the midpoint $(M_{x,i},M_{y,i})$;
    \item the $z_i$ coordinate.}
\end{itemize} 
 {\small
 \item \textbf{Selection}:}
 \vspace{-1mm}
  \begin{itemize}
   {\small \item  Hits in \textbf{one vertical station};
    \item \textbf{Straight line in 3D}: $\geq$4 hits at different $z$ $\rightarrow$ $nhits_{face_i}\geq 1$;
    \item \textbf{Resolution}: $nhits_{panel_i}\leq 3$.}
    \end{itemize} 
 {\small \item \textbf{Reconstruction}: }
 \vspace{-1mm}
        \begin{itemize}
        {\small \item $StrawHit$s $\rightarrow$ 1 $ClusterHit$ (face);
          \item 2 $ClusterHit$s $\rightarrow$ $StereoHit$ (plane);
          \item 2 $StereoHit$s $\rightarrow$ reconstructed track.}
        \end{itemize}

\end{itemize}
\end{column}
\begin{column}{0.5\framewidth}
      \begin{figure}[!h]
      \centering
            \hspace*{-2em}
\includegraphics[width=0.7\columnwidth]{figures/png/Screenshot_20240526_164527.png}
     \label{fig:normalhits}
\end{figure}
\vspace{-5mm}
         \begin{figure}[!h]
      \centering
      \hspace*{-2em}
      \includegraphics[width=0.9\columnwidth]{figures/png/Screenshot_20240810_210144.png}
     \label{fig:normalhits}
\end{figure}
\end{column}
\end{columns}
\end{frame}




\begin{frame}
    \frametitle{Panel illumination pattern and muon directions}
    \begin{columns}
\begin{column}{0.65\framewidth}
\vspace{-8mm}
   \setlength{\leftmargini}{1.2em}
      \begin{itemize}
 {\small
 \item Precise calibration: uniformly distributed hits across the panel;
 \vspace{1.6mm}
\item (Top): muon selections bring to \textbf{non uniform} and spotty panel illumination;
 \vspace{1.6mm}
\item 4/4 overlap areas limited to \textbf{panel edges};
 \vspace{1.6mm}
\item \textbf{Waveform non-linearities};
 \vspace{1.6mm}
\item Selection of \textbf{specific
muon directions};
 \vspace{1.6mm}
\item (Bottom): $m_{yz}=\Delta y /\Delta z$ distribution;
 \vspace{1.6mm}
\item No particles with $m_{yz}\sim 0$ (horizontal) and $m_{yz} \rightarrow \infty$ (vertical);
 \vspace{1.6mm}
\item Mostly with $|m_{yz}| \sim 1$ (\textbf{45° angle});
 \vspace{1.6mm}
\item Muon \textbf{rate} scaled by $1/\cos^2\theta \sim 1/2$ (45° flux) and $1/\sqrt{2}$ (cosmics striking at 45°).}

\end{itemize}
\end{column}
\begin{column}{0.5\framewidth}
\vspace{-2mm}
      \begin{figure}[!h]
      \centering
\includegraphics[width=1.\columnwidth]{figures/pdf/xp_vs_yp_panel0.pdf}
     \label{fig:normalhits}
\end{figure}
\vspace{-5mm}
         \begin{figure}[!h]
      \centering
      \includegraphics[width=1.\columnwidth]{figures/pdf/myz.pdf}
     \label{fig:normalhits}
\end{figure}
\end{column}
\end{columns}
\end{frame}













\begin{frame}
    \frametitle{Longitudinal position reconstruction}

    \begin{columns}
\begin{column}{0.65\framewidth}
\vspace{-6mm}
   \setlength{\leftmargini}{1.2em}
      \begin{itemize}
 {\small
 \item (Top): \textbf{longitudinal reconstructed position} $x_{track}$ (panel frame);
  \vspace{1.2mm}
 \item $x_{track}$: intersection of the reconstructed track with the mean $z_i$ of the panel;
   \vspace{1.2mm}
 \item  \textbf{Bumps} $\rightarrow$ 4/4 requirement consequence;
   \vspace{1.2mm}
\item Different bumps $\rightarrow$ different straws;
  \vspace{1.2mm}
\item (Bottom): longitudinal reconstructed position \textbf{bias} (panel frame);
  \vspace{1.2mm}
 \item $\Delta x = x_{track}-x_{true}$;
   \vspace{1.2mm}
 \item $x_{true}$: MC panel hits mean coordinate;
   \vspace{1.2mm}
\item The bias ranges between [-6,6] cm;
  \vspace{1.2mm}
\item Similar distributions for all panels;
  \vspace{1.2mm}
\item Different straws with different bias; 
  \vspace{1.2mm}
\item $m_{yz}=\frac{\Delta y}{\Delta z}$ not accurately reconstructed.}

\end{itemize}
\end{column}
\begin{column}{0.5\framewidth}
\vspace{-3mm}
      \begin{figure}[!h]
      \centering
\includegraphics[width=1.\columnwidth]{figures/png/x_panel0.png}
     \label{fig:normalhits}
\end{figure}
\vspace{-5mm}
         \begin{figure}[!h]
      \centering
      \includegraphics[width=1.\columnwidth]{figures/png/panel_00_x_bias.png}
     \label{fig:normalhits}
\end{figure}
\end{column}
\end{columns}
\end{frame}






\begin{frame}
    \frametitle{Results}

    \begin{columns}
\begin{column}{0.65\framewidth}
\vspace{-8mm}
   \setlength{\leftmargini}{1.2em}
      \begin{itemize}
 {\small
    \item (Top): 2D distribution of $\Delta x$ vs $x_{true}$;
    \vspace{1.6mm}
    \item Different \textbf{spots} $\rightarrow$ different overlap regions and muon directions;
    \vspace{1.6mm}
    \item (Bottom): $\Delta x$ profile vs $x_{true}$;
    \vspace{1.6mm}
    \item $x_{track}$ reconstruction \textbf{systematics} $\pm$4 cm;\vspace{1.6mm}
    \item $m_{yz}=\frac{\Delta y}{\Delta z}$ not accurately reconstructed;
    \vspace{1.6mm}
    \item Vertical station: \textbf{opposite $y-z$ orientated muons do not cancel out};
    \vspace{1.6mm}
        \item First spots: 90° panels overlap;
        \vspace{1.6mm}
    \item \textbf{Increase of data-taking time};
    \vspace{1.6mm}
\item \textbf{This calibration is expected to become challenging}. 
}

\end{itemize}
\end{column}
\begin{column}{0.5\framewidth}
\vspace{-3mm}
      \begin{figure}[!h]
      \centering
\includegraphics[width=1.\columnwidth]{figures/png/panel_00_x_bias_vs_x.png}
     \label{fig:normalhits}
\end{figure}
\vspace{-5mm}
         \begin{figure}[!h]
      \centering
      \includegraphics[width=1.\columnwidth]{figures/png/panel_00_x_bias_vs_x_prof.png}
     \label{fig:normalhits}
\end{figure}
\end{column}
\end{columns}
\end{frame}







\begin{frame}
    \frametitle{Outline}
    
\begin{itemize}
\item \textcolor{mygray}{Commissioning of the tracker DAQ and FEE:}
\begin{itemize}
         \vspace{2mm}

    \item \textcolor{mygray}{validation of ROC readout;}
             \vspace{1.5mm}

    \item \textcolor{mygray}{study of preamplifiers performance}.
\end{itemize}
\vspace{4mm}
    \item \textcolor{mygray}{First steps towards the station calibration;}
    \vspace{6mm}

    \item Pre-pattern recognition studies;
  \vspace{6mm}
    \item \textcolor{mygray}{Conclusions.}
\end{itemize}
\end{frame}





\begin{frame}
    \frametitle{Introduction}
    \vspace{-3mm}
      \begin{columns}
\begin{column}{1.15\framewidth}
    \setlength{\leftmargini}{1.2em}
    \begin{itemize}
    {\small
        \item Most of \textbf{tracker hits} are $e^-$ and $e^+$ with \textbf{$E<20$ MeV} - \textbf{$\delta$-electrons}:}
        \begin{itemize}
          \vspace{0.8mm}
        {\small
            \item \textbf{Compton scattering}: interaction of $\gamma$s ($n$ capture) with material;
             \vspace{0.8mm}
            \item \textbf{$e^\pm$ pairs}: nuclear recoil processes;
                 \vspace{0.8mm}
            \item \textbf{$\delta$-rays}: interaction of high-energy charged particles with material.}
        \end{itemize}
             \vspace{1.3mm}
       {\small \item Mu2e data: $\geq$7 PB/year $\rightarrow$ \textbf{CPU optimization critical};
            \vspace{1.3mm}
        \item Hit flagging to avoid sending them to pattern recognition;
             \vspace{1.3mm}
        \item Crucial step for several \textbf{physics reasons}:}
               \vspace{0.8mm}
        \begin{itemize}
        {\small
            \item \textbf{CE track reconstruction efficiency};
                 \vspace{0.8mm}
            \item \textbf{Protons}: complementary source to determine muon stopping rate;
                 \vspace{0.8mm}
            \item \textbf{$\bar{p}$ background}: correct background estimate.}
                  \vspace{1.3mm}
    \end{itemize}
    \item {\small\textbf{Data-sample}:}
           \vspace{0.8mm}
    \begin{itemize}
    {\small
        \item \textbf{CE-1BB}: CE signal + pileup (1BB-$1.6 \times 10^7$ protons/pulse);
               \vspace{0.8mm}
        \item \textbf{CE-2BB}: CE signal + pileup (2BB-$3.9 \times 10^7$ protons/pulse);
               \vspace{0.8mm}
     \item \textbf{PBAR-0BB}: $\bar{p}$s and no pileup.}
    \end{itemize}
     \end{itemize}
    \end{column}
        \end{columns}
\end{frame}


\begin{frame}
\frametitle{$\delta$-electrons in Mu2e tracker}
\vspace{-8mm}
    \begin{columns}
        \begin{column}{0.5\framewidth}
            \begin{figure}[!h]
        \centering
        \hspace*{-1em}
        \includegraphics[width =1.1\columnwidth]{figures/png/Screenshot_20240812_152905.png}
       \label{fig:momhits}
\end{figure}
        \end{column}
        \begin{column}{0.5\framewidth}
               \begin{figure}[!h]
        \centering
         \hspace*{-1em}
        \includegraphics[width =0.95\columnwidth]{figures/png/Screenshot_20240815_124710.png}
       \label{fig:momhits}
\end{figure}
        \end{column}
    \end{columns}
    \vspace{-5mm}
    \begin{columns}
        \begin{column}{1.15\framewidth}
    \setlength{\leftmargini}{1.2em}
    \begin{itemize}
    {\small
            \item Momentum distribution of particles making at least one hit in the tracker (Left: \textbf{CE-1BB}, Right: \textbf{PBAR-0BB})     {\footnotesize(blue) all particles, (orange) $\delta$s, (green) $p$ and $D$, (pink) $e^-$, (cyan) $e^+$, (dark green) CE, (black) $\mu$, (beige) $\pi$};
            \vspace{1mm}
            \item (Left): 75\% of hits by $\delta$-electrons (71\% $e^-$, 4\% $e^+$ - Compton scattering);
              \vspace{1mm}
            \item (Left): bump in the $e^+$ distribution ($N(\mu^+ \rightarrow e^+ )/N(\mu^- \rightarrow e^-) \sim 10^{-3}$ for $\mu$ entering the DS and DIO on IPA should be also $10^{-3}$ wrt $\mu^-$ DIF);
              \vspace{1mm}
            \item  (Right): $p\bar{p}$ annihilation in ST $\rightarrow$ multiple tracks with $p \sim 100/200$ MeV/c;
              \vspace{1mm}
            \item Reconstructing these tracks helps constrain background.
          
}
           \end{itemize} 
        \end{column}
      
    \end{columns}
\end{frame}



\begin{frame}
    \frametitle{$\delta$-electrons flagging algorithms}
      \begin{columns}
        \begin{column}{1.15\framewidth}
       {\small Two \textbf{pre-pattern recognition  algorithms} developed in Mu2e Offline:}
    \setlength{\leftmargini}{1.2em}
    \begin{itemize}
    {\small   
    \item \textbf{FlagBkgHits} (FBH). First it finds clusters of hits close in $x-y$ and time and uses an ANN to classify them. Based on $StereoHit$ reconstruction;
    \item \textbf{DeltaFinder} (DF). Search for hit patterns consistent with $\delta$-electron ones:
    }
    \begin{itemize}
        {\small \item $\delta$ segments in each station ($seed$,  3 or 4 hits cluster in space and time);
\item  straws intersection determined $\rightarrow$ center of gravity on $x$-$y$; 
\item $seed$s close in $x$-$y$ and time 
across stations connected ($\delta$ candidate); 
\item $p$ candidates ($seed$s with $\bar{E}_{dep}>$ 3 keV).}
    \end{itemize}
    \end{itemize}
    \end{column}
    \end{columns}
    \vspace{-3mm}
        \begin{columns}
        \begin{column}{0.5\framewidth}
            \begin{figure}[!h]
        \centering
        \hspace*{-2em}
        \includegraphics[width =0.6\columnwidth]{figures/png/Screenshot_20240811_123048.png}
       \label{fig:momhits}
\end{figure}
        \end{column}
        \begin{column}{0.5\framewidth}
               \begin{figure}[!h]
        \centering
         \hspace*{-1em}
        \includegraphics[width =0.75\columnwidth]{figures/png/Screenshot_20240811_115854.png}
       \label{fig:momhits}
\end{figure}
        \end{column}
    \end{columns}
        \vspace{-3mm}
      \begin{columns}
        \begin{column}{1.15\framewidth}
    \setlength{\leftmargini}{1.2em}
    \begin{itemize}
    {\small   
    \item (Left): $\delta$-electrons and CE patterns in the $r-z$ plane;
    \item (Right): A $\delta$ candidate $seed$.}
    \end{itemize}
    \end{column}
    \end{columns}
\end{frame}

\begin{frame}
\frametitle{Performance analysis and comparison}
           \vspace{-0.5mm}
  \begin{columns}
        \begin{column}{1.15\framewidth}
       {\small Two levels of comparison:}
    \setlength{\leftmargini}{1.2em}
    \begin{itemize}
      {\small  \item \textbf{hit-level}: how accurately individual hits are flagged (most direct method);
    \item \textbf{high-level}: reconstruction level comparison (figure of merit: CE tracks).
    \vspace{-1mm}
\item Before comparing: \textbf{proton hit flagging over-efficiency} by \textbf{DF}.}
    \end{itemize}
    \end{column}
    \end{columns}
        \vspace{-2mm}
    \begin{columns}
        \begin{column}{0.5\framewidth}

    \begin{table}[h!]
        \centering
        \renewcommand{\arraystretch}{1.}
        \begin{tabular}{| c | c | c | c | c|} 
        \hline
        &   $f_{p}$ &   $f_{e}$\\
        \hline
        $p$ &  96.0\% & 1.0\% \\
        \hline
        $\mu$ &  5.8\%  & 5.0\%\\
        \hline
        $\pi$ & 2.5\% &  11.2\%\\
        \hline
        \end{tabular}
        \label{tab:0bbpbarbefore}
    \end{table}
        \end{column}
        \begin{column}{0.5\framewidth}
             \begin{figure}[!h]
            \centering
            \includegraphics[width =0.9\columnwidth]{figures/png/Screenshot_20240805_222923.png}
           \label{fig:0pbarbefore}
\end{figure}
        \end{column}
    \end{columns}
    \vspace{-2mm}
    \begin{columns}
        \begin{column}{1.15\framewidth}
    \setlength{\leftmargini}{1.2em}
    \begin{itemize}
      {\small  \item (Left): $f_p$ and $f_e \ \rightarrow$ 
fraction of hits flagged as 
$p$ and $e^-$;
\item (Right): distribution of the total (red) 
        and flagged (green) number of $\mu$ hits as a function of the particle momentum;
\item High $\mu$ and $\pi$ $f_p$ $\rightarrow$ low momentum (higher energy deposition as $p$);
\item $Good$ proton candidate $\rightarrow \ \geq 4$
hits with $E_{dep} >$ 3 keV; 
\item $\epsilon_p$ reduced of 10\%, but $\mu$ and $\pi$ $f_p$ reduced by factor of 2 and 6 $\rightarrow$ next slide.
}
      \end{itemize}
      \end{column}
      \end{columns}
\end{frame}




\begin{frame}
    \frametitle{Hit-level comparison}
    \vspace{-3mm}
    \begin{columns}
    \begin{column}{0.5\framewidth}
    \vspace{-2mm}
        \setlength{\leftmargini}{0.7em}
\begin{itemize}
{\small
    \item (Top Tab): \textbf{PBAR}. $\mu$ and $\pi$ $f_{p,FBH}$ 4x and 3.3x higher;
      \vspace{-0.5mm}
      \item FBH: \textbf{supervised training} with CE+pileup dataset;
      \vspace{-0.5mm}
      \item $\pi$: higher momenta $\rightarrow$ smaller curvature $\rightarrow$ higher $f_e$;
      \vspace{-0.5mm}
      \item (Bottom Tab): \textbf{CE-1BB}. Same results for \textbf{CE-2BB} within 1\%;

    }
\end{itemize}
        \end{column}
         \begin{column}{0.6\framewidth}
        \begin{table}[h!]
        \centering
        \hspace*{-0.5em}
        \renewcommand{\arraystretch}{0.7}
        \begin{tabular}{| c | c | c | c|} 
        \hline
         &  {\scriptsize $f_{p}$ DF} &  {\scriptsize $f_{e}$ FBH} & {\scriptsize $f_{e}$ DF}\\
        \hline
        {\scriptsize $\mu$} &  {\scriptsize 2.7\%}  & {\scriptsize 13.0\%} & {\scriptsize 3.2\%}\\
        \hline
        {\scriptsize $\pi$} & {\scriptsize 0.4\%} & {\scriptsize 23.8\%} & {\scriptsize 7.3\%} \\
        \hline
        \end{tabular}
        \label{tab:0bbpbar}
        \end{table}
        \vspace{-6mm}
        \begin{table}[h!]
    \centering
            \hspace*{-0.5em}
    \renewcommand{\arraystretch}{0.7}
    \begin{tabular}{| l | c | c | c |} 
    \hline
    &    {\scriptsize $f_{p}$ DF} & {\scriptsize $f_{e}$ FBH } & {\scriptsize $f_{e}$ DF} \\
    \hline
    {\scriptsize $e^-$} {\tiny$<$20 MeV/c}      & {\scriptsize 2.5\%}   & {\scriptsize 75.9\%} & {\scriptsize 72.5\%}\\
    \hline
    {\scriptsize $e^-$} {\tiny[20,80] MeV/c}  & {\scriptsize 1.0\%}   & {\scriptsize 50.0\%} & {\scriptsize 27.4\%} \\
    \hline
    {\scriptsize $e^-$} {\tiny[80,110] MeV/c}  & {\scriptsize 0.3\%}  &  {\scriptsize 5.7\%} & {\scriptsize 3.4\%}\\
    \hline
    {\scriptsize $p$}       &         {\scriptsize 83.7\%}   &  & {\scriptsize 1.0\%}\\
    \hline
    {\scriptsize $e^+$} {\tiny$<$20 MeV/c} & {\scriptsize 0.2\%}    &   {\scriptsize 85.5\%}& {\scriptsize 88.5\%}\\
    \hline

    \end{tabular}
    \label{tab:2bbcele}
    \end{table}
  
        \end{column}
    \end{columns}
    \vspace{-4.mm}
          \begin{columns}
      \begin{column}{1.15\framewidth}
    \setlength{\leftmargini}{0.9em}
    \begin{itemize}
      {\small  \item \textbf{70\% more CE hits flagged} as $\delta$-electrons by FBH with respect to DF; 
      \vspace{-0.5mm}
      \item \textbf{No $p$ flagging comparison} (FBH identifies only high $E_{dep}$ particles).
}
      \end{itemize}
      \end{column}
      \end{columns}
          \vspace{-3.7mm}
\begin{columns}
 
    \begin{column}{0.33\framewidth}
        \begin{figure}[!h]
            \centering
            \hspace*{-1em}
            \includegraphics[width =1.2\columnwidth]{figures/png/Screenshot_20240818_155835.png}
           \label{fig:0pbarbefore}
\end{figure}
    \end{column}
      \begin{column}{0.33\framewidth}
        \begin{figure}[!h]
            \centering
                        \hspace*{-0.2em}
            \includegraphics[width =1.\columnwidth]{figures/png/Screenshot_20240820_154854.png}
           \label{fig:0pbarbefore}
\end{figure}
    \end{column}
       \begin{column}{0.4\framewidth}
        \setlength{\leftmargini}{0.5em}
        \begin{itemize}
           {\small 
           \item (Left): $e^-$ and $e^+$ $f_e$ vs momentum (FBH);
           \vspace{-0.5mm}
           \item 1-2 MeV: $\delta$s with 1/2 hits per station (DF>3hits);
                      \vspace{-0.5mm}
           \item >2 MeV: larger $xy$ spread;
                      \vspace{-4.5mm}
           \item (Right): $e^-$ (pink) and $e^+$ (black, no Compton) $E$ distribution ($E<2$ MeV).}
        \end{itemize}
    \end{column}
\end{columns}
   
\end{frame}

\begin{frame}
    \frametitle{High-level comparison}
    \vspace{-5mm}
     \begin{columns}
    \begin{column}{0.5\framewidth}
    \vspace{-2mm}
        \setlength{\leftmargini}{0.7em}
\begin{itemize}
{\small
    \item (Top Tab): \textbf{PBAR}. DF has 22\% advantage in reconstructing 2 tracks (hit-level);
            \vspace{-1mm}
\item 80 MeV/c cut: minimum reconstructable particle $p$ from ST in the tracker;
        \vspace{-1mm}
 \item 90 MeV/c cut: DIO suppression;
    }
\end{itemize}
        \end{column}
         \begin{column}{0.6\framewidth}
        \begin{table}[h!]
        \centering
        \hspace*{-0.5em}
        \renewcommand{\arraystretch}{0.7}
           \begin{tabular}{| c | c | c |} 
            \hline
            {\scriptsize fraction of events} &  {\scriptsize FBH } &  {\scriptsize DF}\\
            \hline
              {\scriptsize $N_{tracks} \geq 2$} &   {\scriptsize 1.8\%} &  {\scriptsize 2.2\%}\\
            \hline
             {\scriptsize $N_{tracks} \geq 2$ \& $p>$80 MeV/c} &  {\scriptsize 1.7\%} &  {\scriptsize 2.1\%}\\
            \hline
             {\scriptsize $N_{tracks} \geq 2$ \& $p>$90 MeV/c } &  {\scriptsize 1.6\%} &  {\scriptsize 2.0\%}\\
            \hline
            \end{tabular}
        \label{tab:0bbpbar}
        \end{table}
        \vspace{-6mm}
        \begin{table}[h!]
    \centering
            \vspace{-1mm}
            \hspace*{-0.5em}
    \renewcommand{\arraystretch}{0.7}
    \begin{tabular}{| c | c | c |} 
    \hline
    & {\scriptsize FBH} & {\scriptsize DF}  \\
    \hline
    {\scriptsize CE events with $N_{tracks}>0$} & {\scriptsize 37.9\%} & {\scriptsize 37.9\%} \\
    \hline
    \end{tabular}
    \label{tab:2bbcele}
    \end{table}
  
        \end{column}
    \end{columns}
    \vspace{-6mm}
 \begin{columns}
        \begin{column}{1.15\framewidth}
    \setlength{\leftmargini}{1.1em}
    \begin{itemize}
    {\small   
    \item (Bottom Tab): \textbf{CE-1BB}. Same CE reconstruction efficiency (the hit-level difference is 1 hit/track);
        \vspace{-0.5mm}
    \item FBH (blue) \& DF (red) reconstructed tracks $p$ distributions (two ranges);
    \vspace{-0.5mm}
    \item FBH flags less proton hits and these hits are sent to pattern recognition.
    }
    \end{itemize}
    \end{column}
    \end{columns}
        \vspace{-3mm}
     \begin{columns}
        \begin{column}{0.5\framewidth}
            \begin{figure}[!h]
        \centering
        \includegraphics[width =0.9\columnwidth]{figures/png/Screenshot_20240820_162125.png}
       \label{fig:momhits}
\end{figure}
        \end{column}
        \begin{column}{0.5\framewidth}
               \begin{figure}[!h]
        \centering
        \includegraphics[width =0.9\columnwidth]{figures/png/Screenshot_20240820_160904.png}
       \label{fig:momhits}
\end{figure}
        \end{column}
    \end{columns}
     
\end{frame}







\begin{frame}
    \frametitle{Outline}
    
\begin{itemize}
\item \textcolor{mygray}{Commissioning of the tracker DAQ and FEE:}
\begin{itemize}
         \vspace{2mm}

    \item \textcolor{mygray}{validation of ROC readout;}
             \vspace{1.5mm}

    \item \textcolor{mygray}{study of preamplifiers performance}.
\end{itemize}
\vspace{4mm}
    \item \textcolor{mygray}{First steps towards the station calibration;}
    \vspace{6mm}

    \item \textcolor{mygray}{Pre-pattern recognition studies;}
  \vspace{6mm}
    \item Conclusions.
\end{itemize}
\end{frame}

\begin{frame}{Conclusions}
\vspace{-4mm}
\begin{columns}
\begin{column}{1.15\framewidth}
 \setlength{\leftmargini}{1.2em}
    \begin{itemize}
\item \textbf{CLFV} processes provide a clean test field for \textbf{NP models};
\item \textbf{Mu2e} is one of the leading experiments and searches for $\mu^- N \rightarrow e^- N$;
\item Mu2e success depends on the performance of the \textbf{tracker};
    \item \textbf{Comprehensive study} from the tracker readout to offline analysis:
   \begin{itemize}
      \vspace{0.7mm}
   \item \textbf{DAQ} and \textbf{FEE} testing:
   \vspace{0.4mm}
      \begin{itemize}
          
     
    \item \textbf{ROC readout valitation} with MC at a level of $10^{-3}$;
    \item \textbf{Preamps performance study}: dead channels, 
cross-talk between the channels, and waveform patterns study.
 \end{itemize}
    \vspace{0.7mm}
 \item \textbf{First steps towards timing calibration}:
  \vspace{0.4mm}
 \begin{itemize}
     \item \textbf{Vertical station orientation} $\rightarrow$ non-uniform panel illumination;
     \item Large \textbf{bias} on $x_{track}$ reconstruction ($\pm$4 cm);
     \item Increase of data-taking time;
 \end{itemize}
   \vspace{0.7mm}
 \item \textbf{Pre-pattern recognition study} to flag $\delta$-electrons:
  \vspace{0.4mm}
  \begin{itemize}
  \item \textbf{Hit-level}: FBH flags 70\% more CE hits, same performance for $\delta$s, no possible $p$ flag comparison, FBH not trained one $\bar{p}$ data sample;
  \item \textbf{High-level}: same recostruction performance for CE signal;
  \item  \textbf{Timing}: 0.13, 0.39 ms/ev (1BB, 2BB) more for DF vs 5 ms/ev expected;
     \item \textbf{Important to improve $TimeCluster$ efficiency}.
 \end{itemize}
   \end{itemize}
    \end{itemize}
    \end{column}
\end{columns}

\end{frame}



\begin{frame} % and our simple frame
    \begin{center}
            \LARGE{\textbf{Thank you for your attention!}}
    \end{center}
\end{frame}
\begin{frame}{Bibliography} % and our simple frame
\begin{columns}
    \begin{column}{1.04\framewidth}
            \setlength{\leftmargini}{1.05em}
    \bibliographystyle{siam}
   {\footnotesize \bibliography{presentazione.bib}}
    \end{column}
\end{columns}

\end{frame}


\begin{frame}
    \frametitle{Beyond the SM}
    \vspace{-3mm}
    \begin{columns}
        \begin{column}{0.5\framewidth}
    \begin{figure}[!h]
        \centering
        \includegraphics[width =0.55\columnwidth]{figures/png/Screenshot_20240218_105920.png}
        \label{fig:susy}
        \end{figure}
    \end{column}
    \begin{column}{0.5\framewidth}
   {\small SUSY contribution to $l_i \rightarrow l_j\gamma$ via $slepton$ }
    \end{column}
\end{columns}
\vspace{-3mm}
        \begin{columns}
            \begin{column}{1.15\framewidth}
            \setlength{\leftmargini}{1.1em}
        \begin{itemize}
    {\footnotesize   \item  \textbf{Supersymmetry}:} 
    \begin{itemize}
        {\footnotesize \item particle with superpartner (different spin), lepton $\rightarrow \ slepton$;
        \item no common mass eigenstate base $\rightarrow$ 
        $slepton$ superposition of flavours;
        \item CLFV suppression $\rightarrow$ separation of the $\nu$ and $W$ masses;
        \item SUSY breaking at electroweak scale ($\sim 10^2$ GeV) $\rightarrow$ observable 
        violation. 
        }
    \end{itemize}
    {\footnotesize  \item \textbf{Two Higgs Doublet model}:}
    \begin{itemize}
        {\footnotesize  \item  two Higgs bosons;
        \item non-zero off-diagonal terms $\rightarrow$ flavour violating Yukawa couplings}
    \end{itemize}
    {\footnotesize  \item \textbf{Leptoquark models}}
    \begin{itemize}
        {\footnotesize   \item LQ has both a baryon and lepton number. 
        \item quark and lepton sectors are unified $\rightarrow$ direct coupling via LQ exchange;
        \item specific CLFV processes are mediated by LQs.}

    \end{itemize}
    {\footnotesize  \item  \textbf{Additional Neutral Gauge Boson}: }
    \begin{itemize}
        {\footnotesize   \item Gauge boson mixes with SM neutral Gauge boson $\rightarrow$
        two mass eigenstates ($Z$, $Z'$). CLFV from off-diagonal terms in neutral
        current couplings to fermions.}

    \end{itemize}


\end{itemize}
\end{column}
\end{columns}
\end{frame}


\begin{frame} % and our simple frame
    \begin{center}
            \LARGE{\textbf{BACKUP SLIDES}}
    \end{center}
\end{frame}
\begin{frame}{Charged Lepton Flavour Violation (CLFV)}
\vspace{-1mm}
\setlength{\leftmargini}{0em}
\begin{itemize}
\item \textbf{EFT} Lagrangian parametrisation (model-indipendent): $\Lambda$ is the effective mass scale and $\kappa$ controls the relative contribution of the dipole moment term and the four fermion term.
\end{itemize}
\begin{equation*}\label{LCF}
\mathscr{L}_{C L F V}= & \frac{m_\mu}{(1+\kappa) \Lambda^2} \bar{\mu}_R \sigma_{\mu \nu} e_L F^{\mu \nu}+ \frac{\kappa}{(1+\kappa) \Lambda^2} \bar{\mu}_L \gamma_\mu e_L\left(\sum_{q=u, d} \bar{q}_L \gamma^\mu \bar{q}_L\right)
\end{equation*}
\vspace{-3mm}
\begin{columns}
\begin{column}{0.5\framewidth}
                \begin{figure}[h]
            \centering
            \hspace*{-6ex}
            \includegraphics[width=0.8\columnwidth]{figures/png/Screenshot_20240913_151711.png}
        \end{figure} 
        \begin{figure}[h]
            \centering
            \hspace*{-6ex}
            \includegraphics[width=0.8\columnwidth]{figures/png/Screenshot_20240913_151719.png}
        \end{figure}  
    \end{column}
    \begin{column}{0.5\framewidth}
        \begin{figure}[h]
            \centering
            \hspace*{-6ex}
            \includegraphics[width=0.85\columnwidth]{figures/png/Screenshot_20240313_120457.png}
        \end{figure}  
    \end{column}
\end{columns}

\end{frame}

\begin{frame}
    \frametitle{muon channels}
    contenuto terza slide
\end{frame}

\begin{frame}
    \frametitle{CLFV experiments}
    contenuto terza slide
\end{frame}
\begin{frame}
    \frametitle{tau channels}
    contenuto terza slide
\end{frame}

\begin{frame}
    \frametitle{Mu2e beamline}
    contenuto terza slide
\end{frame}

\begin{frame}
    \frametitle{Why Al Stopping Target?}
    \vspace{-3mm}
\begin{columns}
 \begin{column}{0.65\framewidth}
 \setlength{\leftmargini}{1.1em}

    \begin{itemize}
   {\small     \item Lower RMC background. Photon endpoint: {\footnotesize $k_{max} = m_\mu c^2 - |E_b| - E_{rec} - \Delta M$ }
        \\
        $\sim$101.9 MeV, $\sim$3.1 MeV below CE; 

\item (Top): quite long lifetime, allowing separation between prompt backgrounds and live window; 


\item (Bottom): DIO endpoint dependence on nucleus type. 
Al $\rightarrow$ high endpoint. Higher-$Z$ nuclei $\rightarrow$ lower endpoint, minimizing background contribution;

\item Conversion $BR$ depends on the ST material. Comparison of conversion $BR$s on different nuclei 
normalized to aluminum $\rightarrow$ dominating 
operator type. Materials with higher $Z$ $\rightarrow$ better model differentiation (Mu2e-II Ti);
\item Available in required size/shape/thickness, low costs and chemically stable.}
    \end{itemize}
\end{column}   
 \begin{column}{0.5\framewidth}
    \begin{figure}[!h]
        \centering
        \includegraphics[width=0.8\columnwidth]{figures/png/lifetime_mu_matter.png}
        \label{fig:muonicatom}
 \end{figure}
       \begin{figure}[!h]
        \centering
        \includegraphics[width=0.8\columnwidth]{figures/png/endopint.png}
        \label{fig:endpoint}
    \label{fig:2imins}
  \end{figure}
\end{column}   
\end{columns}

\end{frame}
\begin{frame}
    \frametitle{Proton pulses and off spill on spill}
\end{frame}

\begin{frame}
    \frametitle{Extinction monitor}
\end{frame}


\begin{frame}
    \frametitle{Cosmic Ray Veto and Stopping Target Monitor}
    \vspace{-2mm}
 \begin{columns}
            \begin{column}{0.55\framewidth}
                     \textbf{Cosmic Ray Veto}:
         \begin{itemize}
         \item \textbf{Active veto}: 4 layers of extruded plastic scintillation counters;
                \item \textbf{Passive shielding}: Al absorbers between each layer;
               \item $\mu$'s signature: 3/4 vetoed.
            \end{itemize}
             \begin{figure}[h]
            \centering
            \includegraphics[width=0.9\columnwidth ]{figures/png/Screenshot_20240706_094517.png}
        \end{figure}
            \end{column}
            \begin{column}{0.55\framewidth}
            \begin{figure}[h]
            \centering
            \includegraphics[width=0.8\columnwidth ]{figures/jpg/Crv_downstream.jpg}
        \end{figure}
        \textbf{Stopping Target Monitor}:
            \begin{itemize}
                \item HPGe and  $ \text{LaBr}_3$ detector $\rightarrow$ number of $\mu$ stopped in ST (10\% precision on $N_\mu$);
                \item It will measure the photons produced by secondary muonic aluminium orbital transitions (347 keV) and nuclear capture (884 keV, 1809 keV).
            
            \end{itemize}
                         
            \end{column}
    \end{columns}
\end{frame}

\begin{frame}
    \frametitle{Drift tubes}
    contenuto terza slide
\end{frame}
\begin{frame}
    \frametitle{Lorentz effect}
    contenuto terza slide
\end{frame}
\begin{frame}
    \frametitle{DAQ system}
\begin{columns}
    \begin{column}{0.65\framewidth}
        \vspace{-3mm}
        \setlength{\leftmargini}{1.2em}
    \begin{itemize} 
   {\footnotesize     \item $Streaming$ readout: digitized and zero-suppressed data; 
        \vspace{-0.5mm}
        \item High data throughput, flexible for analysis; 
        \vspace{-0.5mm}
        \item DAQ run control via RCH, managing a 
        predefined Run Plan; 
        \vspace{-0.5mm}
        \item Active spill defined by RF Zero-Crossing Markers 
        from the Accelerator, synced to \\ 1695 ns proton pulses, defining the EW; 
        \vspace{-0.5mm}
        \item CFO module generates 40 MHz clock, embedding EWMs synced to the system; 
        \vspace{-0.5mm}
        \item CFO distributes the clock and run control packets to DTCs in DAQ servers; 
        \vspace{-0.5mm}
        \item DTCs pass the clock to ROCs and EWMs recovered; 
   
       
   
    }
    \end{itemize}
    \end{column}
    \begin{column}{0.5\framewidth}
        \begin{figure}[!h]
            \centering
            \hspace*{-2em}
            \includegraphics[width =1.1\columnwidth]{figures/png/Screenshot_20240206_144803.png}
            \label{fig:linktodaq}
            \end{figure}
    \end{column}
\end{columns}
\vspace{-1.5mm}
\begin{columns}
    \begin{column}{1.15 \framewidth}
        \setlength{\leftmargini}{1.1em}
        \begin{itemize}
          {\footnotesize 
          \item ROCs use EWMs to separate data from consecutive EWs; 
          \vspace{-0.5mm} 
          \item Data Requests trigger ROC data transmission through DTCs post-event;
        \vspace{-0.5mm} 
          \item EBS routes data from multiple DTCs for online analysis; 
        \vspace{-0.5mm} 
           \item Events logged and transferred to long-term storage (7 PB/year); 
            \vspace{-0.5mm}
            \item DTCs handle slow control data, managed by DCS Host and stored.}
        \end{itemize}
    \end{column}
\end{columns}
\end{frame}
\begin{frame}
    \frametitle{Tracker data format}
    \begin{columns}
        \begin{column}{1.15\framewidth}
    \setlength{\leftmargini}{1.em}

    A \textbf{hit data packet} has
    a fixed length of 256 bits (32 bytes).
    The packet structure is as follows:
    \vspace{2mm}
    \begin{itemize}
        \item 16 bit header (straw index);
        \vspace{1mm}
        \item 16 bit for the TDC left straw end;
        \vspace{1mm}
        \item 16 bit for the TDC right straw end;
        \vspace{1mm}
        \item The ToT (time-over-threshold) values for the two ends 
        of the straw are each stored using 8 bits;
        \vspace{1mm}
        \item The ADC samples require 12 bits each. For each hit, a fixed number of 
        samples (15) is readout;
        \vspace{1mm}
        \item 12 bits are set aside for preprocessing flags.
    \end{itemize}
\end{column}
\end{columns}
    \end{frame}

\begin{frame}
    \frametitle{Additional plots}
    \begin{figure}[!h]
        \centering
        \includegraphics[width =0.8\textwidth]{figures/pdf/figure_00014_nhitsvschannel_roc_simulation_281.pdf}
        \label{fig:anglesinmuon}
    \end{figure}
    \begin{itemize}
        \item Zoom on the last readout channels of the occupancy plot.
    \end{itemize}
\end{frame}


\begin{frame}
    \frametitle{Additional plots}
    \vspace{-2mm}
    \begin{columns}
        \begin{column}{0.5\textwidth}
    \begin{figure}[!h]
        \centering
        \includegraphics[width =0.9\columnwidth]{figures/pdf/charge.pdf}
        \label{fig:anglesinmuon}
    \end{figure}
\end{column}
\begin{column}{0.5\textwidth}
    \begin{figure}[!h]
        \centering
        \includegraphics[width =0.9\columnwidth]{figures/pdf/phch1.pdf}
        \label{fig:anglesinmuon}
    \end{figure}
\end{column}
\end{columns}
\vspace{-2mm}
\begin{figure}[!h]
    \centering
    \includegraphics[width =0.45\textwidth]{figures/pdf/phtmean1.pdf}
    \label{fig:anglesinmuon}
\end{figure}
\vspace{-2mm}
        \begin{columns}
            \begin{column}{1.15\framewidth}
                \setlength{\leftmargini}{1.em}
                \begin{itemize}

      {\small  \item (Left): The (positive) charge distribution of the waveforms (channel 66).;
        \item (Right): 2D distribution of pulse height versus (positive) charge. }
    \end{itemize}
\end{column}
\end{columns}
\end{frame}


\begin{frame}

    \frametitle{Cosmic muons simulation with CRY}
    \begin{figure}[!h]
        \centering
        \includegraphics[width =0.4\textwidth]{figures/png/Screenshot_20240526_140716.png}
        \label{fig:anglesinmuon}
    \end{figure}
        \begin{columns}
            \begin{column}{1.15\framewidth}
                \setlength{\leftmargini}{1.em}

                \begin{itemize}
                   {\small 
                   
                    \item CRY (LANL) was used to generate cosmics (straightforward implementation);
                    \item It simulates protons (1 GeV,100 TeV), at the top of the 
                    atmosphere and generation of muons from the pion decays. 
                    \item It follows Gaisser-Tang model:
                    \\     {\footnotesize               $ \frac{d I}{d E_\mu d \Omega d t d S}=\frac{0.14}{\mathrm{cm}^2 \mathrm{~s} \ \mathrm{sr} }\left( \frac{E_\mu}{\mathrm{GeV}} \left(1+\frac{3.64 \mathrm{GeV}}{E_\mu\left(\cos \theta^*\right)^{1.29}}\right)\right)^{-2.7}\left[\frac{1}{1+\frac{1.1 E_\mu \cos \theta^*}{115 \mathrm{GeV}}}+\frac{0.054}{1+\frac{1.1 E_\mu \cos \theta^*}{850 \mathrm{GeV}}}\right]$};
                     \item cos$\theta^*$ is given by:
                     {\footnotesize  $\cos \theta^*=\sqrt{\frac{(\cos \theta)^2+P_1^2+P_2(\cos \theta)^{P_3}+P_4(\cos \theta)^{P_5}}{1+P_1^2+P_2+P_4}}$
                     \\ with $P_1\sim$0.10, $P_2\sim$-0.07, $P_3\sim$0.96, $P_4\sim$0.04 and $P_5\sim$0.82};
                    \item (Top): $\theta^*$ and $\theta$, zenith angle of muons and 
                   at the muon production point. 
                   }
                \end{itemize}
            \end{column}
            \end{columns}
\end{frame}
\begin{frame}
    \frametitle{How to determine channel-to-channel delays}
$$
            \frac{(t_1 + t_2)}{2} = t_d+t_0+\frac{d_1+d_2}{2}-\cfrac{L}{2v} 
 $$
 \begin{columns}
    \begin{column}{1.15\framewidth}
 \begin{itemize}
    \item $(t_1 + t_2) / 2$ allows to measure the drift time  
    up to an offset common to all channels.

 \end{itemize}

\end{column}
    \end{columns}
\end{frame}
\begin{frame}
    \frametitle{Reconstructed muon direction}
    \begin{figure}[!h]
        \centering
        \includegraphics[width =0.8\textwidth]{figures/png/myz_rec.png}
       \end{figure}
       \begin{columns}
        \begin{column}{1.15\framewidth}
            \begin{itemize}
               {\small \item Lots of muons reconstructed with $m_{yz}\sim 0$ (horizontal);
               \item The true hit position, far from the straws midpoint, 
               results in incorrectly reconstructed tracks' direction on the $y-z$ plane.
                }
            \end{itemize}
        \end{column}
        \end{columns}
\end{frame}




\begin{frame}
    \frametitle{$\delta$-electron sources}
    \vspace{-4mm}
    \begin{columns}
        \begin{column}{0.5\framewidth}
            \begin{figure}[!h]
                \centering
                \includegraphics[width =0.8\columnwidth]{figures/png/Screenshot_20240812_204345.png}
               \end{figure}
        \end{column}
        \begin{column}{0.5\framewidth}
            \begin{figure}[!h]
                \centering
                \includegraphics[width =0.8\columnwidth]{figures/png/Screenshot_20240812_204755.png}
               \end{figure}
        \end{column}
    \end{columns}

    \begin{columns}
        \begin{column}{1.15\framewidth}
            \begin{itemize}
               {\small \item \textbf{Compton scattering}: by $\gamma$s interacting with the detector material. 
               Muon capture $\rightarrow$ neutrons $\rightarrow$ neutron capture $\gamma$ emission ($E_\gamma \sim$MeV). 
               The Compton effect (Left) is the scattering of a photon by a free or quasi-free ($E_\gamma \gg E_B$)
               electron. $e^-/ e^+$ asymmetry. Compton cross section per atom proportional to $Z$;  
               \item \textbf{Pair production} (Right): from nuclear recoil processes. In the Coulomb 
               field of a charge, a photon can convert into an $e^- - e^+$ pair. $Z^2$ dependence.  
               $E_{\gamma} \geq 2m_e c^2 + 2 \frac{m_e^2}{m_{\text{nucleus}}} c^2$;
              
               \item \textbf{Delta rays} (or secondary ionization electrons): generated
               when high-energy charged particles collide with the detector material. 
               A particle collides with shell $e^-$, 
               resulting in significant energy transfers. 
                }
            \end{itemize}
        \end{column}
        \end{columns}
\end{frame}
\begin{frame}
    \frametitle{$\bar{p}$ background in Mu2e}
    \vspace{-4mm}
    \begin{columns}
        \begin{column}{0.5\framewidth}
            \begin{figure}[!h]
                \centering
                \includegraphics[width =0.6\columnwidth]{figures/png/Screenshot_20240926_115031.png}
               \end{figure}
        \end{column}
        \begin{column}{0.5\framewidth}
            \begin{figure}[!h]
                \centering
                \includegraphics[width =0.6\columnwidth]{figures/png/Screenshot_20240926_115050.png}
               \end{figure}
        \end{column}
    \end{columns}
\vspace{-3mm}
    \begin{columns}
    \begin{column}{1.15\framewidth}
        \begin{itemize}
           {\small \item $\bar{p}$s are produced from $pW$ interactions;
           \vspace{-1mm}
           \item $p\bar{p}$ annihilation at ST $\rightarrow$ $e^-$ by $\pi^0\rightarrow \gamma \gamma$ followed by $\gamma$ 
           conversions and $\pi^- \rightarrow \mu^- \bar{\nu}$;
           \vspace{-1mm}
           \item The background cannot be suppressed by cuts on the time window because $\bar{p}$s are slower than other
           beam particles;
           \vspace{-1mm}
           \item There are absorber elements placed in the TS to suppress the $\bar{p}$s;
           \vspace{-1mm}
           \item  $p\bar{p}$ annihilation at ST can give multiple particle tracks with $p \sim$100 MeV/c for each track at much
           higher rate than signal-like;
           \vspace{-1mm}
           \item From MC, it was estimated that the rate of such multi-track
           events is $\times 500$ higher than the rate of events with 1 signal like
           $e^-$;
           \vspace{-1mm}
           \item The analysis aims to reconstruct the multi-track final 
           state events and get an estimate of the CE like events by rescaling the two final states ratio.
            }
        \end{itemize}
    \end{column}
    \end{columns}
\end{frame}

\begin{frame}
    \frametitle{Monte Carlo deposited energy in the tracker}
    \vspace{-2mm}
\begin{figure}[!h]
    \centering
    \includegraphics[width =0.6\textwidth]{figures/png/Screenshot_20240729_151910.png}
   \end{figure}
       \vspace{-3mm}

   \begin{columns}
       \begin{column}{1.15\framewidth}
               \setlength{\leftmargini}{1.1em}
       \begin{itemize}
           
   {\small    \item The Monte Carlo deposited energy 
   distribution in the tracker ($CE-1BB$ data sample);
   \item (Red) CEs, (green) $\delta$-electrons, (blue) protons. 
   \item Peaks and tails $\rightarrow$ saturated waveform; 
   \item Only about 4\% of CE hits have energies above 3.5 keV 
(1\% above 5 keV);
   \item Applying an energy threshold in DF can speed up processing, but impacts algorithm efficiency, especially in $seed$ reconstruction.
        }

       \end{itemize}  
       \end{column}
   \end{columns}
\end{frame}

\begin{frame}
    \frametitle{Mu2e event reconstruction}
    \vspace{-3mm}
\begin{columns}
\begin{column}{0.65\framewidth}
 \setlength{\leftmargini}{1.2em}
 \vspace{-3mm}
\begin{itemize}
{\small\item Mu2e event reconstruction is optimised to
reconstruct single-track events with tracks coming
from the ST;
 \vspace{1.2mm}
\item  Adjacent $StrawHit$s within a panel, which are most likely due to the same particle, are combined 
into a $ComboHit$;
\vspace{1.2mm}
\item $\delta$-electron pre pattern recognition;
\vspace{1.2mm}
\item We cluster the hits within a time window to form
$TimeCluster$s assuming that such hits are made
by the same particle;
\vspace{1.2mm}
\item Hits from $TimeCluster$s are used to form helices;
\vspace{1.2mm}
\item Final parameters of the track are determined by
the Kalman fit.}
\end{itemize}
\end{column}    
\begin{column}{0.5\framewidth}
    \begin{figure}[!h]
        \centering
        \includegraphics[width =0.7\textwidth]{figures/png/Screenshot_20240811_123612.png}
       
        \label{fig:bef}
\end{figure}
     \begin{figure}[!h]
        \centering
        \includegraphics[width =0.7\textwidth]{figures/png/Screenshot_20240811_124245.png}
        
        \label{fig:af}
    \end{figure}
\end{column}
\end{columns}
\end{frame}

\begin{frame}
    \frametitle{Time Clustering}
    \vspace{-3mm}
    \begin{figure}[!h]
        \centering
        \includegraphics[width =0.5\textwidth]{figures/png/Screenshot_20240809_162925.png}
       \end{figure}
       \vspace{-2mm}
    \begin{columns}
        \begin{column}{1.15\framewidth}
            \setlength{\leftmargini}{1.1em}
            {\small \textbf{Time clustering process}:}
            \begin{enumerate}
               {\small \item Combination of at least 3 $ComboHit$s within a specific $time-z$ 
               window (with a 
               20 ns time window and a 5-plane z-window);
               \item $Chunk$s are created;
               \item  Every potential pair of chunks 
               within a certain time 
               proximity is tested together, 
               and the pair, that minimizes 
               the $\chi^2/ndof$ when the hits 
               are fit to a linear line, is combined;
               \item Procedure repeated until no further combinations 
               yield a $\chi^2/ndof$ below a set 
               threshold.
                }
            \end{enumerate}
        \end{column}
        \end{columns}

\end{frame}

\begin{frame}
    \frametitle{Time Clustering development}
    \vspace{-4mm}
    \begin{columns}
        \begin{column}{0.5\framewidth}
            \begin{figure}[!h]
                \centering
                \includegraphics[width =\columnwidth]{figures/png/Screenshot_20240926_095738.png}
               \end{figure}
        \end{column}
        \begin{column}{0.5\framewidth}
            \begin{figure}[!h]
                \centering
                \includegraphics[width =\columnwidth]{figures/png/Screenshot_20240926_095542.png}
               \end{figure}
        \end{column}
    \end{columns}

\begin{columns}
\begin{column}{1.15\framewidth}
    \begin{itemize}
       {\small \item There is a well defined class of events where the effects of hit flaggers get washed out in the reconstruction by the time clustering algorithm;
        \item Example:
        \begin{itemize}
            \item DF: hits from one particle divided in two different time clusters;
            \item FBH: not flagged particle hits are used by the time clusterer to $connect$ particle hits that are used in the reconstruction. That is why the track is reconstructed in this case.
        \end{itemize}
        
        \item Improving the cluster finder and the pattern recognition could increase the track reconstruction performance.
        }
    \end{itemize}
\end{column}
\end{columns}
\end{frame}

\end{document}

