%%%%%%%%%%%%%%%%%%%%%%%%%%%%%%%%%%%%%%%%%%%%%%%%%%%%%%%%%%%
%
% Copyright 2022 by Isac Pasianotto
%
% This file may be distributed and/or modified
%
% 1. under the LaTeX Project Public License and/or
% 2. under the GNU Public License.
%
%%%%%%%%%%%%%%%%%%%%%%%%%%%%%%%%%%%%%%%%%%%%%%%%%%%%%%%%%%%

%% 	Variabili di tipo "color": 

\definecolor{bluUnits100}{rgb}{0.16,0.22,0.36} 
\definecolor{bluUnits80}{rgb}{0.22,0.3,0.51}
\definecolor{bluUnits70}{rgb}{0.25,0.35,0.58}
\definecolor{bluUnits50}{rgb}{0.41,0.51,0.74}
\definecolor{bluUnits40}{rgb}{0.56,0.63,0.81}
\definecolor{bluUnits25}{rgb}{0.78,0.82,0.9}
\definecolor{bluUnits10}{rgb}{0.93,0.94,0.96}
\definecolor{grey}{rgb}{0.3686, 0.5255, 0.6235} 
\definecolor{coolblack}{rgb}{0.0, 0.18, 0.39}
\definecolor{UBCblue}{rgb}{0.0, 0.18, 0.39} % UBC Blue (primary)
\definecolor{UBCgrey}{rgb}{0.3686, 0.5255, 0.6235} % UBC Grey (secondary)
%%	Palette di colori 

\setbeamercolor{palette primary}{bg=UBCblue,fg=white}
\setbeamercolor{palette secondary}{bg=UBCblue,fg=white}
\setbeamercolor{palette tertiary}{bg=UBCblue,fg=white}
\setbeamercolor{palette quaternary}{bg=UBCblue,fg=white}
\setbeamercolor{palette light primary}{bg=bluUnits25,fg=bluUnits100}
\setbeamercolor{palette titleframe}{bg=bluUnits10, fg=bluUnits80}

\setbeamercolor{palette primary}{bg=UBCblue,fg=white}
\setbeamercolor{palette secondary}{bg=UBCblue,fg=white}
\setbeamercolor{palette tertiary}{bg=UBCblue,fg=white}
\setbeamercolor{palette quaternary}{bg=UBCblue,fg=white}
\setbeamercolor{author in sidebar}{fg=black!20!white}
\setbeamercolor{title in sidebar}{fg=white}
\setbeamercolor{section in sidebar}{fg=white}
\setbeamercolor{structure}{fg=UBCblue} % itemize, enumerate, etc
\setbeamercolor{sidebar}{bg=UBCblue,fg=white}

		%%%%%%%%%%%%%%%%%%%%%%%%%%%%%%%%%%
		%% Impostazioni generali slide  %%
		%%%%%%%%%%%%%%%%%%%%%%%%%%%%%%%%%%

%%	Setta l'immagine da mettere come sfondo, riducendone l'opacità
%\usebackgroundtemplate{\tikz\node[opacity=0.1]{\includegraphics[height=\frameheight]{\LogoFiligrana}};}

%%	Elenchi puntati, numerati, etc.

\setbeamercolor{structure}{fg=UBCblue}
\setbeamertemplate{enumerate item}[circle]
\setbeamertemplate{itemize subitem}[ball]
% Valutare a secoda del contesto se sostituire con 
% \setbeamertemplate{items}[circle]
\setbeamercolor{alerted text}{fg=white}

%% 	Colore delle scritte nella presentazione

%\setbeamercolor{normal text}{fg=bluUnits100,bg=white}
\setbeamercolor{normal text}{fg=UBCblue}

%% 	Settaggio della linea in alto (headline)
\setbeamertemplate{navigation symbols}{}


%%	Settaggio riga in basso (footline) 

\setbeamertemplate{footline}
{
  \leavevmode%
  \hbox{%
  \begin{beamercolorbox}[wd=.333333\paperwidth,ht=2.25ex,dp=1ex,center]{author in head/foot}%
    \usebeamerfont{author in head/foot}\insertshortauthor
  \end{beamercolorbox}%
  \begin{beamercolorbox}[wd=.333333\paperwidth,ht=2.25ex,dp=1ex,center]{title in head/foot}%
   \usebeamerfont{institute in head/foot}\insertshortinstitute
  \end{beamercolorbox}%
  \begin{beamercolorbox}[wd=.333333\paperwidth,ht=2.25ex,dp=1ex,center]{date in head/foot}%
  \hspace*{12ex}
    \usebeamerfont{date in head/foot}\insertshortdate\hspace*{12ex}
    \insertframenumber{}/\inserttotalframenumber
  \end{beamercolorbox}}%
    
  \vskip0pt%
}
%%	Settaggio tittoli delle slide  

\setbeamertemplate{frametitle}{
	\begin{beamercolorbox}[wd=\paperwidth,ht=2.75ex,dp=1ex,left]{palette titleframe}
		\hspace*{2ex}\textbf{\insertframetitle}
	\end{beamercolorbox}
}


		%%%%%%%%%%%%%%%%%%%%%%%%%%%%%%%
		%% Impostazioni Prima Slide  %%
		%%%%%%%%%%%%%%%%%%%%%%%%%%%%%%%
	
	
	
	
		
\def\setTitlestyleDissertation{
	
	\defbeamertemplate*{title page}{customized}[1][]{
		
		%  Commentare il seguente ambiente {center} e decommentare {flushright} quello successivo in caso
		%	si voglia usare solo il logo dell'UNI
		
		\begin{center}
			\begin{multicols}{2}
				\includegraphics[width=0.34\textwidth]{\LogoDipartimento}
			\columnbreak
				\includegraphics[width=0.28\textwidth]{\LogoUniversita}		
			\end{multicols}
		\end{center}
	
		%	\begin{flushright}
		%		\includegraphics[width=0.45\textwidth]{\LogoUniversita}	
		%	\end{flushright}
	
		\smallskip
		
		\begin{center}		
			\usebeamerfont{title}\textbf{\inserttitle}\par
			\usebeamerfont{subtitle}\usebeamercolor[fg]{subtitle}\insertsubtitle\par
			\bigskip		
			
			%% Il seguente layout dentro l'ambiente multicols serve per le tesi.
			
			\begin{multicols}{2}
                    \begin{tabular}{c}
					\usebeamerfont{normal text}{\candidatoLabel} \\
					\usebeamerfont{author}{\insertauthor}
				\end{tabular}
    				\columnbreak
				\begin{tabular}{c}
					\usebeamerfont{normal text}{\relatoreLabel} \\
					\usebeamerfont{author}{\relatore} \\
					\usebeamerfont{author}{\correlatore}
						
				\end{tabular}					
				
			\end{multicols}
		
			\par
			
			\bigskip  	% --> nel caso di relatore e basta
			%\smallskip 	% --> nel caso di relatore + correlatore
			
			%\insertinstitute\par
			
			%\bigskip	% --> nel caso di relatore e basta
			%\smallskip	% --> nel caso di relatore + correlatore
			
			\usebeamerfont{date}\insertdate\par
			
			\bigskip	% --> nel caso di relatore e basta
			%\smallskip	% --> nel caso di relatore + correlatore
		\end{center}
	}
}
